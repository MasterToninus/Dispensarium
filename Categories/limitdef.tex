%% A tentative, eye-catchy, one-page definition of category (I would like to replicate the page frome my handwritten notes.
\documentclass[preview]{standalone}

\usepackage{amsmath}
\usepackage{verbatim}
\usepackage[italian,english]{babel}
\usepackage[utf8]{inputenc}

\usepackage[basic,cat]{../Math-Symbols-List/toninus-math-symbols}
\usepackage{../Latex-Theorem/theoremtemplate}
\usepackage{./visualcat}

\begin{document}
% For every picture that defines or uses external nodes, you'll have to
% apply the 'remember picture' style. To avoid some typing, we'll apply
% the style to all pictures.
\tikzstyle{every picture}+=[remember picture]

% By default all math in TikZ nodes are set in inline mode. Change this to
% displaystyle so that we don't get small fractions.
\everymath{\displaystyle}



Let's begin from the dry abstract nonsense:
\begin{definition}[Diagram on $\cat$]
	Functor $D: \cat[I] \rightarrow \cat$ from a small category $\cat[I]$ to $\cat[C]$.
\end{definition}
Basically it is  a graph composed of objects and arrows in $\cat$.

% first column
\begin{minipage}[t]{0.5\textwidth}
	\begin{definition}[Cone over diagram $D$]
		\begin{displaymath}
		 (
		 \tikz[baseline]{
		 			\node[fill=blue!20,anchor=base] (t1)
		            {$ V$};
		        } 
			,
			 \tikz[baseline]{
		            \node[fill=blue!20,anchor=base] (t2)
		            {$ \{\nu_i:V \rightarrow D_i\}_{i\in \cat[I]} $};
			}
		)
		\end{displaymath}
		\tikzstyle{na} = [baseline=-.5ex]
		\begin{itemize}
		    \item \tikz[na] \node[coordinate] (n1) {};
		    	Object $\in \cat$
		    \item Family of morphisms of $\cat$
		        \tikz[na]\node [coordinate] (n2) {};
		\end{itemize}
		
		% Now it's time to draw some edges between the global nodes. Note that we
		% have to apply the 'overlay' style.
		\begin{tikzpicture}[overlay]
		        \path[->] (n1) edge [bend left] (t1);
		        \path[->] (n2) edge [bend right] (t2);
		        %\path[->] (n3) edge [out=0, in=-90] (t3);
		\end{tikzpicture}



	\end{definition}



\end{minipage}
\vrule{}
\begin{minipage}[t]{0.5\textwidth}
	\begin{definition}[CoCone over diagram $D$]
		\begin{displaymath}
		 (
		 \tikz[baseline]{
		 			\node[fill=blue!20,anchor=base] (t1)
		            {$ W$};
		        } 
			,
			 \tikz[baseline]{
		            \node[fill=blue!20,anchor=base] (t2)
		            {$ \{\omega_i:D_i \rightarrow W\}_{i\in \cat[I]} $};
			}
		)
		\end{displaymath}
		\tikzstyle{na} = [baseline=-.5ex]
		\begin{itemize}
		    \item \tikz[na] \node[coordinate] (n1) {};
		    	Object $\in \cat$
		    \item Family of morphisms of $\cat$
		        \tikz[na]\node [coordinate] (n2) {};
		\end{itemize}
		
		% Now it's time to draw some edges between the global nodes. Note that we
		% have to apply the 'overlay' style.
		\begin{tikzpicture}[overlay]
		        \path[->] (n1) edge [bend left] (t1);
		        \path[->] (n2) edge [bend right] (t2);
		        %\path[->] (n3) edge [out=0, in=-90] (t3);
		\end{tikzpicture}



	\end{definition}



\end{minipage}


\end{document}