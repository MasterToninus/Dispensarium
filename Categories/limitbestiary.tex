%% A tentative, eye-catchy, one-page definition of category (I would like to replicate the page frome my handwritten notes.
\documentclass[preview]{standalone}

\usepackage{amsmath}
\usepackage{verbatim}
\usepackage[italian,english]{babel}
\usepackage[utf8]{inputenc}

\usepackage[basic,cat]{../Math-Symbols-List/toninus-math-symbols}
\usepackage{../Latex-Theorem/theoremtemplate}
\usepackage{./visualcat}
\usepackage{pdflscape}




\begin{document}
The best way to grasp the meaning (...) to understand what is a limit it is through examples. The model category is $\cat[Set]$ which turn out to be complete and co-complete.

\begin{definition}[Terminal Object]
	\begin{tikzcd}
		\textcolor{blue}{X} \ar[d,green] \\
		\textcolor{red}{T} 
	\end{tikzcd}
\end{definition}
%
\begin{definition}[Initial Object]
	\begin{tikzcd}
		\textcolor{red}{I} \ar[d,green] \\
		\textcolor{blue}{X} 
	\end{tikzcd}
\end{definition}
%
\begin{definition}[Product]
	\begin{tikzcd}
		&\textcolor{blue}{C} \ar[ddl, bend right, blue] \ar[ddr, bend left, blue] \ar[d,green] 
		\ar[ddl,phantom,green, "\commute"] \ar[ddr,phantom,green, "\commute"]& \\
		&\textcolor{red}{A \times B} \ar[dl,red] \ar[dr,red]& \\
		A & & B \\
	\end{tikzcd}
\end{definition}
%
\begin{definition}[Co-Product]
	\begin{tikzcd}
		A \ar[dr,red] \ar[ddr, bend right, blue] & & B\ar[dl,red] \ar[ddl, bend left, blue]\\
		&\textcolor{red}{A \times B}  \ar[d,green] & \\
		&\textcolor{blue}{C} \ar[uul,phantom,green, "\commute"] \ar[uur,phantom,green, "\commute"]& 
	\end{tikzcd}
\end{definition}
%
\begin{definition}[Equalizer]

\end{definition}
%
\begin{definition}[Co-Equalizer]

\end{definition}
%
\begin{definition}[Pull-Back]

\end{definition}
%
\begin{definition}[Push-Out]

\end{definition}
%
So, a limit is a concept pertaing to a specific configuration of objects and arrows (the image of the diagram $D$ in $\cat$).
\\
We reserve a special name for categories such that limit exists for all diagram of a certain shape regardless of the specificity of vertex and arrows (i.e. depending only on the shape of the diagrams).
%
\begin{notation}
	\begin{displaymath}
		D: \cat[I] \rightarrow	\cat
	\end{displaymath}
	A diagram is \emph{finite} if $\Obj(\cat[I])$ is a discrete set.
	\\
	Limits on a finite diagram are called \emph{finite limits}.
\end{notation}
%
\begin{definition}[(finitely) (Co-)Complete Category]
	$\cat$ \St all (finite) (Co-)Limit exists.
\end{definition}
%
\begin{proposition}
T.f.s.a.e.:
\begin{itemize}
	\item $\cat$ has all (finite) limits / (finite) colimits
	\item $\cat$ has all (finite) products and equalizers / (finite) co-products and co-equalizers
	\item $cat$ has all (binary) pullbacks and terminal object / (binary) pushout and initial object
\end{itemize}

\end{proposition}



\end{document}