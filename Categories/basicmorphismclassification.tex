%% A tentative, eye-catchy, one-page definition of category (I would like to replicate the page frome my handwritten notes.
\documentclass[preview]{standalone}

\usepackage{amsmath}
\usepackage{verbatim}
\usepackage[italian,english]{babel}
\usepackage[utf8]{inputenc}

\usepackage[basic,cat]{../Math-Symbols-List/toninus-math-symbols}
\usepackage{../Latex-Theorem/theoremtemplate}
\usepackage{./visualcat}

\usepackage{tikz}
\def\tria{
	\tikz[baseline=.1ex]{
		\fill (0,-0.5ex) coordinate (A);
		\fill (0,2ex) coordinate (B);
		\fill (2.25ex,0.75ex) coordinate (C);
		\draw[green] (0.75ex,0.75ex) node {$\commute$};
		\draw[blue] (A) -- (B);
		\draw[green] (A) -- (C) -- (B);
	}
}

\def\trib{
	\tikz[baseline=.1ex]{
		\fill (0,-0.5ex) coordinate (A);
		\fill (0,2ex) coordinate (B);
		\fill (2.25ex,0.75ex) coordinate (C);
		\draw[blue] (0.75ex,0.75ex) node {$\commute$};
		\draw[blue] (A) -- (B);
		\draw[blue] (A) -- (C) -- (B);
	}
}

\def\cotria{
	\tikz[baseline=.1ex]{
		\fill (0,-0.5ex) coordinate (A);
		\fill (0,2ex) coordinate (B);
		\fill (-2.25ex,0.75ex) coordinate (C);
		\draw[green] (-0.75ex,0.75ex) node {$\commute$};
		\draw[blue] (A) -- (B);
		\draw[green] (A) -- (C) -- (B);
	}
}

\def\cotrib{
	\tikz[baseline=.1ex]{
		\fill (0,-0.5ex) coordinate (A);
		\fill (0,2ex) coordinate (B);
		\fill (-2.25ex,0.75ex) coordinate (C);
		\draw[blue] (-0.75ex,0.75ex) node {$\commute$};
		\draw[blue] (A) -- (B);
		\draw[blue] (A) -- (C) -- (B);
	}
}

\begin{document} 
% first column
\begin{minipage}[t]{0.5\textwidth}
	\begin{definition}[Monomorphism (mono or monic arrow)]
		\emph{"left cancellative arrow"}
		$\textcolor{red}{\bulk} s.t. \textcolor{blue}{\forall \bulk} \quad \tria \Rightarrow \trib$
		
		\begin{tikzcd}
			 \textcolor{blue}{B} \ar[dd,blue,equal] \ar[dr,blue,"g_1"] \arrow[rrd, bend left=30, "fg_1",green] & & \\
			\phantom. & \textcolor{red}{C}  \ar[r,red,"f",tail]& \textcolor{red}{D}  \\
			\textcolor{blue}{B}  \ar[ur,blue,"g_2"] \arrow[rru, bend right=30, "fg_2",green]& & 
		\end{tikzcd}
		\footnote{$(f g_1 = f g_2 ) \Rightarrow ( g_1 = g_2 )$}
	\end{definition}
 
\end{minipage}
%second column
\begin{minipage}[t]{0.5\textwidth}
	\begin{definition}[Epimorphism (epi or epic arrow)]
		\emph{"right cancellative arrow"}
		$\textcolor{red}{\bulk} s.t. \textcolor{blue}{\forall \bulk} \quad \cotria \Rightarrow \cotrib$
		
		\begin{tikzcd}
			& & \textcolor{blue}{E} \ar[dd,blue,equal] \\
			\textcolor{red}{C} \ar[r,red,"f",tail] \arrow[rru, bend left=30, "h_1 f",green] \arrow[rrd, bend right=30, "h_2 f",green]& \textcolor{red}{D} \ar[ur,blue,"h_1"] \ar[dr,blue,"h_2"] & \\
			& & \textcolor{blue}{E}
		\end{tikzcd}
		\footnote{$(h_1 f = h_2 f ) \Rightarrow ( h_1 = h_2 )$}
	\end{definition}
\end{minipage}






\end{document}
