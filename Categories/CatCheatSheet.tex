%% A tentative, eye-catchy, one-page definition of category (I would like to replicate the page frome my handwritten notes.

\documentclass[a4paper,12pt,fleqn]{scrartcl}  %Per La Stampa
\usepackage[a4paper, margin=2cm]{geometry}

\usepackage{amsmath}
\usepackage[basic,cat]{../Math-Symbols-List/toninus-math-symbols}
\usepackage[]{standalone}
\usepackage{commath}

\usepackage[italian,english]{babel}
\usepackage[utf8]{inputenc}

\usepackage{standalone}

\usepackage{graphicx}
\usepackage{hyperref}

\usepackage{../Latex-Theorem/theoremtemplate}

\usepackage[toc,page]{appendix}


%%__________________________________
%% To transform in a package.
%%__________________________________
%%
	\newcommand{\catdef}[3]{
		\begin{displaymath}
		#1 \leadsto
		\begin{cases}
			Obj = #2\\
			Arr = #3
		\end{cases}
		\end{displaymath}
	}
	\newcommand{\smalltikzcd}[2]{
		\begin{tikzpicture}[baseline= (a).base]
			\node[scale=#1] (a) at (0,0){
				\begin{tikzcd} #2 \end{tikzcd}
			};
		\end{tikzpicture}
	}


	\newcommand{\id}{\textrm{id}}

%Diagrammatic
\usepackage{tikz-cd}
\usepackage{xcolor}
\usepackage{fdsymbol}
	% bulk of arrows and objects
	\newcommand{\bulk}{\blacklozenge}
	% such that
	\newcommand{\St}{\textrm{s.t.}}
	% id est
	\newcommand{\ie}{\textrm{\emph{i.e.}}}
	% commuting
	\newcommand{\commute}{\circledequal}%\circlearrowleft



\begin{document} 
\begin{abstract}
	The preliminary material for my talk on the category of diffeological spaces.
\end{abstract}


\begin{appendices}

%------------------------------------------------------------------------------------BASIC ---------------------------------
\section{Basic definitions}
	%% A tentative, eye-catchy, one-page definition of category (I would like to replicate the page frome my handwritten notes.
\documentclass[preview]{standalone}

\usepackage{amsmath}
\usepackage{verbatim}
\usepackage[italian,english]{babel}
\usepackage[utf8]{inputenc}

\usepackage[basic,cat]{../Math-Symbols-List/toninus-math-symbols}
\usepackage{../Latex-Theorem/theoremtemplate}
\usepackage{./visualcat}


\begin{document} 

\begin{definition}[Category ]
Category is a tuple $($data,structure,axioms$)$.
\begin{itemize}
\item \emph{data} 
		\catdef{\cat}
			{\Obj(\cat), \textrm{a collection of items called \emph{objects of $\cat$}}}
			{\hom_{\cat} (A,B), \textrm{a parametrized family of collections of items called \emph{morphisms} or \emph{arrows}} \forall A,B \in  \Obj(\cat)}	

\item Structure:
	\begin{displaymath}
		\circ : \hom_{\cat}(A,B) \times \hom_{\cat}(B,C) \rightarrow \hom_{\cat} (A,C)
	\end{displaymath}
	Diagrammatically: 
	$\forall \bulk \textcolor{red}{\exists \bulk} \textcolor{blue}{\St \commute}$\\
	\begin{tikzcd}
		A \arrow[d, "f"] \arrow[dd, bend left=60, "gf = f \circ g",red] 	& \\
		B \arrow[d, "g"] \arrow[r, phantom,"\commute",, very near start,blue] 	& \phantom.\\
		C 																& 
	\end{tikzcd}	
		(This is more properly a diagrammatic definition of  $\commute$)

\item Axioms:
	\begin{itemize}
		\item \emph{Identity axiom}
			$\forall \bulk \textcolor{red}{\exists \bulk} \textcolor{blue}{\St \commute}$\\
			\begin{tikzcd}
				A \arrow[rr, "\id_A",red] \arrow[rrdd, "f", bend right]& & A \arrow[dd, "f"] \arrow[rrdd, "f", bend left] & & \\
				 & |[blue]|\commute & & |[blue]|\commute & \\
				 & &  B \arrow[rr, "\id_B",red] & & B
			\end{tikzcd}
		\item \emph{associativity axiom}
			$\forall \bulk \textcolor{green}{h (gf) = (hg) f}$			\\
			\begin{tikzcd}
				A \arrow[r,"f"] \arrow[rr,"gf",bend left=40, blue] \arrow[rrr,"h(gf)",bend left=80, green] \arrow[rrr,"(hg)f",bend right=80, green] &
				B \arrow[r,"g"] \arrow[rr,"hg",bend right=40, blue] &
				C \arrow[r,"h"] &				
				D  
			\end{tikzcd}
	\end{itemize}
	
\end{itemize}
\end{definition}
\end{document}

We can not delve here on the foundational issues related to this definitions. Let us simple look at it a triple composed of \emph{data}, \emph{structure}, \emph{axiom}.

One of these problem is the problem of "size". We use the term "collections" when we want to remain vague about the axiomatic framework of sets considered.
We use "mapping" when we want to consider correspondences between collections.

Words like "set" and "function" are reserved to objects and morhpisms in the set category.

\paragraph{How to compare cat?}
	%% A tentative, eye-catchy, one-page definition of category (I would like to replicate the page frome my handwritten notes.
\documentclass[preview]{standalone}

\usepackage{amsmath}
\usepackage{verbatim}
\usepackage[italian,english]{babel}
\usepackage[utf8]{inputenc}

\usepackage[basic,cat]{../Math-Symbols-List/toninus-math-symbols}
\usepackage{../Latex-Theorem/theoremtemplate}
\usepackage{./visualcat}


\begin{document} 

\begin{definition}[Functor]
	A functor $F: \cat[C] \rightarrow \cat[D] $ it's a pair of correpondences 
	\begin{displaymath}
		F: \Obj(\cat[C]) \rightarrow \Obj(\cat[D])
	\end{displaymath}
	\begin{displaymath}
		F_{A,\, B} : \hom_{\cat[C]}(A,B) \rightarrow \hom_{\cat[D]}(F(A),F(B))		\qquad \forall A,B \in \Obj(\cat[C]) 
	\end{displaymath}
	
	\begin{tikzcd}
		A \arrow[d,"f"] & & F(A) \arrow[d,"F(f)",blue, shift left=1ex] \\
		B								& & F(B) \arrow[u,"F(f)",green, shift left=1ex]
	\end{tikzcd}
	(\textcolor{green}{controvariant}) (\textcolor{blue}{covariant})
	
	\St
	\begin{itemize}
		\item \emph{identity axiom}
			$\forall \bulk  \textcolor{blue}{\therefore \commute}$
			\begin{tikzcd}
				A \arrow[d,"\id_A"] & & F(A)\arrow[d, phantom,"\commute",,blue]  \arrow[d, bend right=60, "F(\id_A)"'] \arrow[d, bend left=60, "\id_{F(A)}"] \\
				A & & F(A)
			\end{tikzcd}
		\item \emph{associativity axiom}
			$\forall \bulk  \textcolor{blue}{\therefore \commute}$
			\begin{tikzcd}
		A \arrow[d, "f"] \arrow[dd, bend right=60, "gf "] 	& & F(A) \arrow[d, "F(f)"] \arrow[dd, bend left=60, "F(gf) "] &	 \\
		B \arrow[d, "g"] & &  F(B) \arrow[d, "F(g)"] \arrow[r, phantom,"\commute",, very near start,blue]  & \phantom.\\
		C 	& & F(C)	&
			\end{tikzcd}
	\end{itemize}
\end{definition}
\end{document}


	\subparagraph{Basic Classification:}
	%% A tentative, eye-catchy, one-page definition of category (I would like to replicate the page frome my handwritten notes.
\documentclass[preview]{standalone}

\usepackage{amsmath}
\usepackage{verbatim}
\usepackage[italian,english]{babel}
\usepackage[utf8]{inputenc}

\usepackage[basic, diffgeo,cat]{../Math-Symbols-List/toninus-math-symbols}
\usepackage{../Latex-Theorem/theoremtemplate}
\usepackage{./visualcat}


\begin{document} 

\begin{definition}[Faithful Functor]
	$F_{A,\; B} $ is injective. \ie : \\
	$\forall \bulk  \textcolor{blue}{s.t. \bulk} \textcolor{red}{\Rightarrow \bulk}$
	\begin{tikzcd}
		A \arrow[d, bend right=60, "f "'] \arrow[d, bend left=60, "g "] \arrow[d, phantom,"\commute",,red] & & 
		F(A) \arrow[d, bend right=60, "F(f) "'] \arrow[d, bend left=60, "F(g) "] \arrow[d, phantom,"\commute",,blue] \\
		B & & F(B)
	\end{tikzcd}
\end{definition}
%
\begin{definition}[Full Functor]
	$F_{A,\; B} $ is surjective. \ie : \\
	$\forall \bulk  \textcolor{blue}{\exists \bulk} \textcolor{red}{\St \bulk}$
	\begin{tikzcd}
		A \arrow[d, "f ",blue]  & &  
		F(A) \arrow[d, bend right=60, "F(f) "', blue] \arrow[d, bend left=60, "\eta "] \arrow[d, phantom,"\commute",,red] \\
		B & & F(B)
	\end{tikzcd}
\end{definition}
%
\begin{definition}[Costant Functor]
	$ \bigtriangleup_u : \cat \rightarrow \cat[D] $ s.t.
	\begin{tikzcd}
		A \arrow[d, "f "]  & &  
		\bigtriangleup_u(A) = u \arrow[d, "\bigtriangleup_u(f) = \id_u"']  \\
		B & & \bigtriangleup_u(A) = u
	\end{tikzcd}
\end{definition}

\end{document}


\subparagraph{How to compare Functors?}
	%% A tentative, eye-catchy, one-page definition of category (I would like to replicate the page frome my handwritten notes.
\documentclass[preview]{standalone}

\usepackage{amsmath}
\usepackage{verbatim}
\usepackage[italian,english]{babel}
\usepackage[utf8]{inputenc}

\usepackage[basic,cat]{../Math-Symbols-List/toninus-math-symbols}
\usepackage{../Latex-Theorem/theoremtemplate}
\usepackage{./visualcat}


\begin{document} 

\begin{definition}[Natural transformation]
	\begin{tikzcd}
		\mu: F \xrightarrow[]{\cdot} G & = &
		\cat[C] \arrow[r, bend left=50, "F"{name=U, above}]
		\arrow[r, bend right=50, "G"{name=D, below}]
		& \cat[D]
		\arrow[shorten <=10pt,shorten >=10pt,Rightarrow, from=U, to=D, "\mu"]
	\end{tikzcd}
	\\
	Is a collection of morphisms 
	\begin{displaymath}
		\left\lbrace \mu_X : F(X) \rightarrow G(X) \: \vert \: X \in \Obj(\cat) \right\rbrace \subset \Arr(\cat[D])
	\end{displaymath} 
	S.t. \\
	$\forall \bulk  \textcolor{red}{\exists \bulk} \textcolor{blue}{\St \bulk}$\\
	\begin{tikzcd}
		X \arrow[d, "f "]  & & 
		F(X) \arrow[d, "F(f) "'] \arrow[r, "\eta_X ", red ] \arrow[dr,"\commute",phantom,blue] & G(X) \arrow[d, "G(f) "] \\
		Y & & F(Y)\arrow[r, "\eta_Y ", red ] & G(Y)
	\end{tikzcd}
\end{definition}

\end{document}


%------------------------------------------------------------------------------------Limits ---------------------------------
\section{Limits}
Let's begin from the dry abstract nonsense:
\begin{definition}[Diagram on $\cat$]
	Functor $D: \cat[I] \rightarrow \cat$ from a small category $\cat[I]$ to $\cat[C]$.
\end{definition}
Basically it is  a graph composed of objects and arrows in $\cat$.

\begin{minipage}[t]{0.5\textwidth}
	\begin{definition}[Cone on $D$]
		\documentclass[preview]{standalone}
\usepackage{tikz}
\usepackage{amsmath}
\usepackage{verbatim}
\usetikzlibrary{arrows,shapes}

\usepackage{./visualcat}

\begin{document}
% For every picture that defines or uses external nodes, you'll have to
% apply the 'remember picture' style. To avoid some typing, we'll apply
% the style to all pictures.
\tikzstyle{every picture}+=[remember picture]

% By default all math in TikZ nodes are set in inline mode. Change this to
% displaystyle so that we don't get small fractions.
\everymath{\displaystyle}

\begin{displaymath}
 (,
 \tikz[baseline]{
            \node[fill=blue!20,anchor=base] (t1)
            {$ V1$};
        } 
	,
	 \tikz[baseline]{
            \node[fill=blue!20,anchor=base] (t2)
            {$ V2$};
	}
	,
	 \tikz[baseline]{
            \node[fill=blue!20,anchor=base] (t3)
            {$ V3$};
	}	 
)
\end{displaymath}
\tikzstyle{na} = [baseline=-.5ex]
\begin{itemize}
    \item Termine 1
        \tikz[na] \node[coordinate] (n1) {};
    \item Termine 1
        \tikz[na]\node [coordinate] (n2) {};
    \item Termine 1
        \tikz[na]\node [coordinate] (n3) {};
\end{itemize}

% Now it's time to draw some edges between the global nodes. Note that we
% have to apply the 'overlay' style.
\begin{tikzpicture}[overlay]
        \path[->] (n1) edge [bend left] (t1);
        \path[->] (n2) edge [bend right] (t2);
        \path[->] (n3) edge [out=0, in=-90] (t3);
\end{tikzpicture}

\end{document}
	\end{definition}
\end{minipage}
\begin{minipage}[t]{0.5\textwidth}
	\begin{definition}[Co-Cone on $D$]
	
	\end{definition}
\end{minipage}

\section{Classification of Morphism}

\section{Subobjects}


\section{Cartesian Closedness}



\section{Ethernal TODO}
	\documentclass[preview]{standalone}

\usepackage{amsmath}
\usepackage{verbatim}

\usepackage[basic,cat]{../Math-Symbols-List/toninus-math-symbols}
\usepackage{./visualcat}

\begin{document}
	%Category
	\begin{displaymath}
		\frac{\cat}{B}
	\end{displaymath}
	% Objects
	\begin{displaymath}
		Obj = \{ \pi: E \rightarrow M \} = \{ \pi \in \Arr(\cat) \vert \cod(\pi)=B\}
	\end{displaymath}
	% Arrows
	\begin{displaymath}
	 \hom_{\dfrac{\cat}{B}}(\pi_1,\pi_2) = \left\{\phi \in \hom_{\cat}(\cod(\pi_1),\cod(\pi_2)) \vert 
		\begin{tikzpicture}[baseline= (a).base]
			\node[scale=.75] (a) at (0,0){
				\begin{tikzcd} E \ar[rr,"\phi"] \ar[dr,"\pi_1"] &  \phantom.\arrow[d, phantom,"\commute"] & E' \ar[dl,"\pi_2"] \\ & M & \end{tikzcd}
			};
		\end{tikzpicture}
	\right\}
	\end{displaymath}

\end{document}


\end{appendices}
			\nocite{*}
			\bibliographystyle{plain}
			\bibliography{biblio}

\end{document}
