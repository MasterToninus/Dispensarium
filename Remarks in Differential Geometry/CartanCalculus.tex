\documentclass[a4paper,12pt]{scrartcl}
\usepackage{pdflscape}
\usepackage{tabularx}
\usepackage{array}
\setlength{\extrarowheight}{6mm}
\usepackage{geometry}
 \geometry{
 a4paper,
 total={185mm,277mm},
 left=5mm,
 right=5mm,
 top=5mm,
 bottom=5mm,
 }
 
\usepackage[subpreambles=true]{standalone}
 
\usepackage{amssymb}
\usepackage{amsmath}
\usepackage{amsfonts}

\renewcommand*{\thefootnote}{\alph{footnote}}



\providecommand{\lie}{\mathcal{L}}
\providecommand{\dext}{\textrm{d}}


%https://tex.stackexchange.com/questions/48980/whole-page-table-with-tabularx

\begin{document}
  \begin{landscape}
    \thispagestyle{empty}
    \noindent
    	    Suppose $M$ is a smooth manifold, denote by $\Omega(M)$ the algebra of differential forms on $M$ and by $\mathfrak{X}(M)$ the $C^\infty(M)$-module of vector fields.  \\
    \vspace{5mm}
    \begin{tabularx}{\linewidth}{|c|X|c|c|c|c|}
      \hline
     	  & $f \in C^\infty(M)$ & $\dext x^\mu \in \Omega^1(M)$ & $\omega^{(k)} \wedge \beta$ & $T_1 \otimes T_2$ & $\partial_\mu \in \mathfrak{X}(M)$ \\
      \hline
      	$\dext$ & $\left(\partial_\nu f \right) \: \dext ^\nu$ & 0 & $\left( \dext \omega \right) \wedge \beta + (-)^k \omega \wedge \left( \dext\beta \right) $ & - & - \\
%      	$\lie_X$ & $X(f)$ & $\lie_X \dext x^\mu = \dext \left(X^a \partial_a x^\mu\right) =  \dext \left(X^a \delta_a^\mu \right) = \dext\left(X^\mu\right) = \partial_\nu X^\mu \dext x^\nu$ & $\left( \lie_X \omega \right) \wedge \beta + \omega \wedge \left(\lie_X\beta \right)$ & $\left(\lie_X T_1\right) \otimes T_2 + T_1 \otimes \left(\lie_X T_2 \right)$ & $\lie_X \partial_\mu = [X, \partial_mu]$ \\
      	$\lie_X$ & $X(f) = x^\mu \partial_\mu f$ & $\lie_X \dext x^\mu = \dext\left(X^\mu\right) = \partial_\nu X^\mu \dext x^\nu$ & $\left( \lie_X \omega \right) \wedge \beta + \omega \wedge \left(\lie_X\beta \right)$ & $\left(\lie_X T_1\right) \otimes T_2 + T_1 \otimes \left(\lie_X T_2 \right)$ & $\lie_X \partial_\mu = [X, \partial_\mu]$ \\
      	$\iota_X$  & $0$ & $\iota_X \dext x^\mu = \dext x^\mu (X) = X^\mu$ & $\left( \iota_X \omega \right) \wedge \beta + (-)^k \omega \wedge \left( \iota_X\beta \right) $ & $\left(\iota_X T_1\right) \otimes T_2 + T_1 \otimes \left(\iota_X T_2 \right)$  & $0$ \\
      	$g^\ast$  & $g^\ast \left(f\right) = g \circ f $ & $ g^\ast \left(\dext x^\mu \right) = \dext\left(x^\mu \circ f \right)$& $g^\ast\left(\omega\right) \wedge g^\ast \left( \beta \right)$ & $g^\ast \left( T_1\right) \otimes g^\ast \left( T_2\right)$ & $(g^{-1})_\ast \partial_\mu$ \quad \footnote{$g$ has to be a diffeo} \\ %& & & & & \\
      \hline
    \end{tabularx}

	
	\begin{minipage}{0.46 \linewidth}
		\fbox{
		  \parbox{\textwidth}{
			    The \emph{Cartan calculus} consists of the following three \emph{graded derivations} on $\Omega(M)$
				\begin{itemize}\itemsep0em 
					\item the \emph{exterior derivative} $d$;
					\item the space of \emph{Lie derivative operators} $\lie_X$, where $X \in \mathfrak{X}(M)$;
					\item the space of \emph{contraction operators} $\iota_X$, where $X \in \mathfrak{X}(M)$.
				\end{itemize}
				\begin{center}
						\includestandalone[scale=0.80]{cartancalculusdiagram}	
				\end{center}
				
				Together with the following identities:
				\begin{align}
					\dext^2 &= 0, \label{cartfirst}\\
					\dext \lie_X - \lie_X d &= 0, \\
					\dext \iota_X + \iota_X d &= \lie_X, \label{magic}\\
					\lie_X \lie_Y - \lie_Y \lie_X &= \lie_{[X,Y]}, \\
					\lie_X \iota_Y - \iota_Y \lie_X &= \iota_{[X,Y]},\\
					\iota_X \iota_Y + \iota_Y \iota_X &= 0, \label{cartlast}
				\end{align}
		  }
		}

	\end{minipage}
	\hspace{1cm}
	\begin{minipage}{0.46 \linewidth}
		\fbox{
		  \parbox{\textwidth}{
				A \emph{graded derivation} of $\Omega(M)$ is a degree $k$ linear operator $A$ on $\Omega(M)$ such that
				\begin{equation}
					A (\omega \wedge \eta) = A(\omega) \wedge \eta + (-1)^{kp} \omega \wedge A(\eta) \qquad \forall \omega \in \Omega^k(M) , \; \eta \in \Omega^\cdot(M)
				\end{equation}
		  }
		}
		\vfill
		\fbox{
		  \parbox{\textwidth}{
				\begin{displaymath}
			    	\Omega(M) = \left(\bigoplus_{k=0}^m \Omega^k(M), \wedge \right) \qquad \textrm{\emph{Grassmann Algebra} on M}
			    \end{displaymath}
			    is a graded commutative graded algebra over ring $C^\infty(M)$.
		  }
		}
		\fbox{
		  \parbox{\textwidth}{
				\begin{displaymath}
			    	\mathcal{T}(M) = \left(\bigoplus_{l,k=0}^\infty
			    	 T^k_l(M), 
			    	 \otimes \right) 
			    	 \qquad \textrm{\emph{Tensor Algebra} on M}
			    \end{displaymath}
			    is a commutative graded algebra over ring $C^\infty(M)$.
		  }
		}


	\end{minipage}
	
	\begin{minipage}{0.46 \linewidth}
\newpage
			\fbox{
		  \parbox{\textwidth}{
		  	How to compute brackets:
		  	\begin{align}
		  		\left[\partial_i, \partial_j\right] = 0 \\
		  		\left[f \partial_i , g \partial_j\right]= f \cdot (\partial_i g)\cdot\partial_j - g\cdot(\partial_j f )\cdot\partial_i \\
		  		\left[f X , g Y \right] = f\cdot g\cdot[X,Y] + f\cdot X(g)\cdot Y - g\cdot Y(f)\cdot X  \\
		  		\left[X ,Y\right] = \left( X^i \partial_i Y^j - Y^i\partial_i X^j \right) \partial_j		  				
		  	\end{align}
		  	$(\mathfrak{X}(M) , [-,-])$ form a Lie algebra over $\mathbb{R}$:
		  	\begin{align*}
		  		[X,Y] = -[Y,X] \\
		  		[aX + bY, Z] = a[X,Z] + b[Y,Z] \\
		  		[[X,Y],Z] + [[Y,Z],X] + [[Z,X],Y] = 0 
		  	\end{align*}
		  }
		}
	\end{minipage}
	\hspace{1cm}
	\begin{minipage}{0.46 \linewidth}
 		TODO: glossary:
 			\begin{itemize}
 				\item Graded vector space
 				\item Graded algebra
 				\item linear operator of degree k
 				\item substitute $\partial_i f $ with $\dfrac{\partial f}{\partial x^i}$ in order to avoid confusion between natural basis vector and usual partial derivative operator acting on function$f$
 			\end{itemize}
	\end{minipage} 
    
  \end{landscape}
\end{document}