\documentclass[a4paper,12pt]{scrartcl}    %attenzione! uso scratctl che permette sottotitolo

%Per le Figure
\usepackage[english]{babel}
\usepackage{graphicx}

%simboli matematici strani quali unione disgiunta
\usepackage{amssymb}

%Scrivere Sotto i simboli
\usepackage{amsmath}

%Diagrammi Commutativi
\usepackage{tikz}
\usetikzlibrary{matrix}

%Il simbolo di Identità
\usepackage{dsfont}

%Per riflettere i simboli...
\usepackage{graphicx}


%link iNTERNET
\usepackage{hyperref}

%Enumerate with letters
\usepackage{enumerate}

%Slash over letter
\usepackage{cancel}

%Usare bibiliografia bibtex
%\bibliographystyle{plain}

%Danger sign
\usepackage{fourier}

%:=
\usepackage{mathtools}



%Common symbols
%Common math symbols
	%Number fields
		\newcommand{\Real}{\mathbb{R}}
		\newcommand{\Natural}{\mathbb{N}}
		\newcommand{\Relative}{\mathbb{Z}}
		\newcommand{\Rational}{\mathbb{Q}}
		\newcommand{\Complex}{\mathbb{C}}
	
%equality lingo
	%must be equal
		\newcommand{\mbeq}{\overset{!}{=}} 

% function
	%Domain
		\newcommand{\dom}{\mathrm{dom}}
	%Range
		\newcommand{\ran}{\mathrm{ran}}
	

% Set Theory
	% Power set (insieme delle parti
		\newcommand{\PowerSet}{\mathcal{P}}

%Differential Geometry
	% Atlas
		\newcommand{\Atlas}{\mathcal{A}}
	%support
		\newcommand{\supp}{\textrm{supp}}

	
	
%Category Theory
	%Mor set http://ncatlab.org/nlab/show/morphism
%		\newcommand{\hom}{\textrm{hom}}

%Geometric Lagrangian Mechanics
	% Kinematic Configurations
		\newcommand{\Conf}{\mathtt{C}}
	%Solutions Space
		\newcommand{\Sol}{\mathtt{Sol}}
	%Lagrangian class
		\newcommand{\Lag}{\mathsf{Lag}}
	%Lagrangiana
		\newcommand{\Lagrangian}{\mathcal{L}}
	%Data
		\newcommand{\Data}{\mathsf{Data}}
	%unique solution map
		\newcommand{\SolMap}{\mathbf{s}}
		
		
		
%Peierls (per non sbagliare più)
		\newcommand{\Pei}{Peierls}


%Temporaneo, Aggiunta della mia classe teorem... Deve diventare un pacchetto!
\input{../Latex-Theorem/TheoremTemplateToninus.tex}




\begin{document}

%  Titolo
	\title{AQFT mathematical preliminaries}
	\author{Tony}
	\date{\today}
\maketitle

%  Indice
\tableofcontents

\begin{abstract}
		(sono ripetizioni inutili per la tesi, sono informazioni che si ritrovano ovunque... sono informazioni adatte al "knowledge base")
\end{abstract}

%%%%%%%%%%%%%%%%%%%%%%%%%%%%%%%%%%%%%%%%%%%%%%%%%%%%%%%%%%%%%%%%%%%%%%%%%%%%%%%%%%%%%%%%%%%%%%%%%%%%%%%%%%%%%%%%%%%%%%%%%%%%%%%%%%%%%%
\newpage
\section{Globally Hyperbolic SpaceTimes}
	Recurring definitions in general Relativity (excluding the general smooth manifold prolegomena).

	\begin{definition}[SpaceTime]
		A quadruple $(M, g, \mathfrak{o}, \mathfrak{t})$ such that:
		\begin{itemize}
			\item $(M,g)$ is a time-orientable n-dimensional manifold $(n>2)$
			\item $\mathfrak{o}$ is a choice of orientation
			\item $\mathfrak{t}$ is a choice of time-orientation
		\end{itemize}
	\end{definition}

	\begin{definition}[Lorentzian Manifold]
		A pair $(M, g)$ such that:
		\begin{itemize}
			\item $M$ is a n-dimensional $(n\geq2)$, Hausdorff, second countable, connected, orientable smooth manifold.
			\item $g$ is a Lorentzian metric.
		\end{itemize}
	\end{definition}
			
	\begin{definition}[Metric]
		A function on the bundle product of $TM$ with itself: $$g: TM \times_M TM \rightarrow \Real$$ such that the restriction on each fiber $$g_p: T_pM \times T_pM \rightarrow \Real $$ is a non-degenerate bilinear form.
	\end{definition}
	
	\begin{notationfix}
		A Pseudo-riemmanian manifold $(M,g)$ is called:
		 \begin{itemize}
		 	\item \emph{Riemmanian} if the sign of $g$ is positive definite.%, \emph{Pseudo-Riemman} otherwise.
		 	\item \emph{Lorentzian} if the signature is $(+, -, \ldots,- )$ or equivalently $(-,+,\ldots,+)$.
		 \end{itemize}
	\end{notationfix}

	\begin{observation}[Causal Structure]
		If a smooth manifold is endowed with a Lorentzian metric of signature $(+, -, \ldots, -)$ then the tangent vectors at each point in the manifold can be classed into three different types. 
		\begin{notationfix}
			$\forall p \in M, \quad \forall X \in T_pM$, the vector is:
			\begin{itemize}
				\item \emph{time-like} if $g(X,X)>0$.
				\item \emph{light-like} if $g(X,X)=0$.
				\item \emph{space-like} if $g(X,X)<0$.
			\end{itemize}
		\end{notationfix}
	\end{observation}

	\begin{observation}[Local Time Orientability]
		$\forall p\in M$ the timelike tangent vectors in $p$ can be divided into two equivalence classes taking
		\begin{displaymath}
			X \sim Y \; \textrm{iff} \; g(X,Y)>0 \qquad \forall X,Y \in T^\textrm{time-like}_pM:
		\end{displaymath}
		We can (arbitrarily) call one of these equivalence classes "future-directed" and call the other "past-directed". Physically this designation of the two classes of future- and past-directed timelike vectors corresponds to a choice of an arrow of time at the point. 
		\\
		The future- and past-directed designations can be extended to null vectors at a point by continuity.
	\end{observation}
	
	\begin{definition}[Time-orientation]
		A global tangent vector field  $\mathfrak{t}\in \Gamma^\infty(TM)$ over the Lorenzian manifold $M$ such that:
		\begin{itemize}
			\item $\supp(\mathfrak{t}) = M$
			\item $\mathfrak{t}(p)$ is time-like $\forall p \in M$.
		\end{itemize}
	\end{definition}
	\begin{observation}
		The fixing of a time-orientation is equivalent to a consistent smooth choice of a local time-direction.
	\end{observation}	
	
	\begin{definition}[Time-Orientable Lorentzian Manifold]
		A Lorentzian Manifold $(M,g)$ such that exist at least one time-orientation $\mathfrak{t}\in \Gamma^\infty(TM)$.
	\end{definition}

	\begin{notationfix}
		Consider a piece-wise smooth curve $\gamma: \Real\supset I \rightarrow M$ is called:
		\begin{itemize}
			\item \emph{time-like} (resp. light-like, space-like) iff $\dot{\gamma}(p)$ is time-like (resp. light-like, space-like) $\forall p \in M$.
			\item \emph{causal} iff $\dot{\gamma}(p)$ is nowhere spacelike.
			\item \emph{future directed} (resp. past directed) iff is causal and  $\dot{\gamma}(p)$ is future (resp. past) directed $\forall p \in M$.
		\end{itemize}
	\end{notationfix}

	\begin{definition}[Chronological $\substack{\textrm{ future}\\ \textrm{past } }$ of a point]
		Are two subset related to the generic point $p	\in M$:
		\begin{displaymath}
			\mathbf{I}_M^\pm(p) \coloneqq \big\{ q \in M \big\vert \; \exists \gamma \in C^\infty\big((0,1), M\big)\;  \textrm{\footnotesize time-like } \substack{\textrm{future}\\ \textrm{past} } -\textrm{\footnotesize directed }:\; \gamma(0)=p,\; \gamma(1)=q  \big\}
		\end{displaymath}
	\end{definition}
	
	\begin{definition}[Causal $\substack{\textrm{ future}\\ \textrm{past } } $ of a point]
		Are two subset related to the generic point $p	\in M$:
		\begin{displaymath}
			\mathbf{J}_M^\pm(p) \coloneqq \big\{ q \in M \big\vert \; \exists \gamma \in C^\infty\big((0,1), M\big)\; \textrm{\footnotesize causal } \substack{\textrm{future}\\ \textrm{past} } -\textrm{\footnotesize directed }:\; \gamma(0)=p,\; \gamma(1)=q  \big\}
		\end{displaymath}		
	\end{definition}

	\begin{notationfix}
		Former concept can be naturally extended to subset $A \subset M$:
			\begin{itemize}
				\item $\mathbf{I}_M^\pm(A) = \bigcup_{p\in A} \mathbf{I}_M^\pm(p) $
				\item $\mathbf{J}_M^\pm(A) = \bigcup_{p\in A} \mathbf{J}_M^\pm(p) $
			\end{itemize}
	\end{notationfix}

	\begin{definition}[Achronal Set]
		Subset $\Sigma \subset M$ such that every inextensible timelike curve intersect $\Sigma$ at most once.
	\end{definition}

	\begin{definition}[$\substack{\textrm{ future}\\ \textrm{past } } $ Domain of dependence of an Achronal set]
		The two subset related to the generic achornal set $\Sigma \subset M$:
		\begin{displaymath}		
			\mathbf{D}_M^\pm(\Sigma) \coloneqq \big\{ q \in M \big\vert \; \forall \gamma \substack{\textrm{ past}\\ \textrm{ future} }\textrm{\footnotesize inextensible causal curve passing through }q : \; \gamma(I) \cap \Sigma \neq \emptyset  \big\}
		\end{displaymath}		
	\end{definition}

	\begin{notationfix}
		$\mathbf{D}_M(\Sigma)  \coloneqq \mathbf{D}_M^+(\Sigma) \cup \mathbf{D}_M^-(\Sigma)$ is called \emph{total domain of dependence}.
	\end{notationfix}

	\begin{definition}[Cauchy Surface]
		Is a subset $\Sigma \subset M$ such that:
		\begin{itemize}
			\item closed
			\item achronal
			\item $\mathbf{D}_M(\Sigma) \equiv M$
		\end{itemize}
	\end{definition}

	The term "hypersurface" is not used by chance:	
	\begin{proposition}
		Every Cauchy surface $\Sigma$ is a three dimensional, embedded, $C^0$ submanifold of $M$
	\end{proposition}
	\begin{proof}
		See Wald (general relativity) teo 8.3.1
	\end{proof}
	

		\begin{Warning}
		\danger copiato da \cite{primer}
		From a physical point of view, we are interested in those spacetimes which allow to set a well-posed
initial value problem for hyperbolic partial differential equations, such as the scalar D’Alambert wave equation, to quote the simplest, yet most important example. In particular we need to ensure that the spacetime we consider possesses at least one distinguished codimension 1 hypersurface on which we can assign the initial data needed to construct a solution of such an equation.	
		\end{Warning}	
		
		\begin{definition}[Globally-Hyperbolic SpaceTime]\label{Def:GHSP}
			 Spacetime $M$ such that there exists at least one \emph{Cauchy Surface}
		\end{definition}
		
		According to Definition \ref{Def:GHSP}, only the existence of a single Cauchy hypersurface is guaranteed. 
		This is slightly disturbing since there is no reason a priori why an initial value hypersurface for a certain partial differential equation should be distinguished. 
		This quandary has been overcome proving that, if a spacetime $(M,g)$ is globally hyperbolic, then there exists a foliation of $M$ by Cauchy surfaces:
		
			\begin{theorem}[Globally hyperbolic  space characterization]\label{Teo:GHSC_character}
				Let $(M,g)$ be any time-oriented spacetime. The following two statements are equivalent:
				\begin{itemize}
					\item $(M,g)$ is globally hyperbolic.
					\item $(M,g)$ is isometric to $ \Real \times \Sigma $ 
						endowed with the line element $ds^2 = \beta dt^2 - h_t$ 
						where $t : \Real \times \sigma \rightarrow \Real$ is the projection on the first factor, 
						$\beta$ is a smooth and strictly positive function on $\Real \times \Sigma$ 
						and $t \mapsto h_t , t \in \Real$, yields a one-parameter family of smooth Riemmanian metrics.\\
						Furthermore, for all $t\in \Real$, $\{t\}\times \Sigma$ is an (n−1)-dimensional, spacelike, smooth Cauchy surface in M.
				\end{itemize}
			\end{theorem}
			\begin{proof}
				Teo 3.17 in J. K. Beem, P. E. Ehrlich and K. L. Easley, Global Lorentzian Geometry.
				\\
				section 1.3 in \cite{barwav}
			\end{proof}

		To conclude this section, we introduce some terms which will be often used in the following in order to
specify the support properties of the sections of a vector bundle with base a globally hyperbolic spacetime.
		\begin{notationfix}
		Let $M$ be a globally hyperbolic spacetime and $E=(E,\pi,M;V)$ a vector bundle of typical fiber $V$.
		We denote:
		\begin{itemize}
			\item $\Gamma_0(E)$ the space of \emph{compactly supported} smooth sections.
			\item $\Gamma_{sc}(E)$  the space of \emph{spacelike compact} smooth sections.\\
				$\big(\; f\in \Gamma_{sc}(E)$ if there exists a compact subset $K \subset M$  such that $\supp f \subset \mathbf{J}_M(K)$. $\big)$
			\item  $\Gamma_{fc}(E) $ the space of \emph{future- compact} smooth sections.\\
				$\big(\; f\in \Gamma_{fc}(E) $ if  $\supp(f) \cap  \mathbf{J}^+_M(K)$ is compact for all $p\in M$.$\big)$
			\item  $\Gamma_{pc}(E) $ the space of \emph{past- compact} smooth sections.\\
				$\big(\; f\in \Gamma_{pc}(E) $ if  $\supp(f) \cap  \mathbf{J}^-_M(K)$ is compact for all $p\in M$.$\big)$
			\item $\Gamma_{tc}(E) \coloneqq \Gamma_{pc}(E) \cap \Gamma_{fc}(E) $ the space of \emph{timelike compact} smooth sections.
		\end{itemize}
		\end{notationfix}
		
		This class of manifolds includes most of the physically interesting examples, e.g.: Minkowski spacetimes, Friedman-Robertson-Walker solutions, Kerr family. \cite{advances}
					
			A trivial example:		
			\begin{example}
				Trivially, the real line $\Real$ is a globally hyperbolic manifold.
				\\
				Each point $x\in \Real$ represent a proper Cauchy surfaces which realize the trivial foliation $\Real \simeq 1\times \Real $ required by theorem \ref{Teo:GHSC_character}
			\end{example}
		
		
%%%%%%%%%%%%%%%%%%%%%%%%%%%%%%%%%%%%%%%%%%%%%%%%%%%%%%%%%%%%%%%%%%%%%%%%%%%%%%%%%%%%%%%%%%%%%%%%%%%%%%%%%%%%%%%%%%%%%%%%%%%%%%%%%%%%%%
\section{Linear Differential Operator}
Basic Definition in L.P.D.O. on smooth vector sections.
\\
Consider $F=F(M,\pi,V), F'=F'(M,\pi',V')$ two linear vector bundle over $M$ with different typical fiber
	\begin{definition}[Linear Partial Differential operator \footnotesize( of order at most $s\in \Natural_0$)]
		Linear map $L:\Gamma(F)\rightarrow \Gamma(F')$ such that:
		\\
		$\forall p \in M$ exists:
		\begin{itemize}
			\item $(U, \phi)$ local chart on $M$.
			\item $(U, \chi)$ local trivialization of $F$
			\item $(U, \chi')$ local trivialization of $F'$
		\end{itemize}
		for which:
		\begin{displaymath}
			L \big(\sigma \big\vert_U\big) = \sum_{\vert \alpha \vert \leq s} A_\alpha \partial^\alpha \sigma \qquad \forall \sigma \in \Gamma(M)
		\end{displaymath}
	\end{definition}

	\begin{remark}
	(multi-index notation)
	\\
	A multi-index is a natural valued finite dimensional vector $\alpha = ( \alpha_0, \ldots, \alpha_n-1) \in \Natural^n_0$ with $n<\infty$.
	\\
	On $\Real^n$ a general differential operator can be identified by a multi-index:
	\begin{displaymath}
		\partial^\alpha = \prod_{\mu = 0}^{n-1} \partial_\mu ^{\alpha_\mu}
	\end{displaymath}
	(Until the Schwartz theorem holds, the order of derivation is irrelevant.)
	\\
	The order of the multi-index is defined as:
	\begin{displaymath}
		\vert \alpha \vert \coloneqq \sum_{\mu=0}^{n-1} \alpha_\mu
	\end{displaymath}
	\end{remark}

	?????????????????????
	\begin{proposition}[Existence and uniqueness for the Cauchy Problem]
		\begin{hypothesis}
			\begin{itemize}
				\item $\mathbf{M} = (M,g,\mathfrak{o},\mathfrak{t}) $a globally hyperbolic space-time.
				\item $\Sigma \subset M$ a spacelike cauchy surface with future-pointing unit normal vector field $\vec{n}$.
				\item 
			\end{itemize}
		\end{hypothesis}
	\begin{thesis}

	\end{thesis}
	\end{proposition}
	\begin{observation}
	"Green-hyperbolic operators are not necessarily hyperbolic in any PDE-sense and that they cannot be characterized in general by well-posedness of a Cauchy problem.	" \cite{Terlaky2010} \cite{Bar2010}
	\\
	However the existence and uniqueness can be proved for the large class of the \emph{Normally-Hyperbolic Operators}.
	
	\end{observation}

		\begin{Warning}
		(ADVANCES)\\
		Globally hyperbolic spacetimes play a pivotal role, not only because they do not allow for pathological situations, such as closed causal curves, but also because they are the natural playground for classical and quantum fields on curved backgrounds. 
		More precisely, the dynamics of most (if not all) models, we are interested in, is either ruled by or closely related to wave-like equations. Also motivated by physics, we want to construct the associated space of solutions by solving an initial value problem. 
		To this end we need to be able to select both an hypersurface on which to assign initial data and to identify an evolution direction. In view of Theorem 1, globally hyperbolic spacetimes appear to be indeed a natural choice. 
		Goal of this section will be to summarize the main definitions and the key properties of the class of partial differential equations, useful to discuss the models that we shall introduce in the next sections. 
		Since this is an overkilled topic, we do not wish to make any claim of being complete and we recommend to an interested reader to consult more specialized books and papers for more details.
		\end{Warning}
		
%-_-_-_-_-_-_-_-_-_-_-_-_-_-_-_-_-_-_-_-_-_-_-_-_-_-_-_-_-_-_-_-_-_-_-_-_-_-_-_-_-_-_-_-_-_-_-_-_-_-_-_-_-_-_-_
%-_-_-_-_-_-_-_-_-_-_-_-_-_-_-_-_-_-_-_-_-_-_-_-_-_-_-_-_-_-_-_-_-_-_-_-_-_-_-_-_-_-_-_-_-_-_-_-_-_-_-_-_-_-_-_
\begin{thebibliography}{100}

\bibitem{advances} Benini, M. and Dappiaggi, C. in Advances in AQFT 1–49

\bibitem{primer} Benini, M., Dappiaggi, C. and Hack, T.-P. Quantum Field Theory on Curved Backgrounds — a Primer. Int. J. Mod. Phys. A 28, 1330023 (2013).

\bibitem{barwav} Bar, C., Ginoux, N. and Pfaeffle, F. Wave Equations on Lorentzian Manifolds and Quantization. (2008).

\end{thebibliography}		


\end{document}