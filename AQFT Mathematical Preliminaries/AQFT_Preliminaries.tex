\documentclass[a4paper,12pt]{scrartcl}    %attenzione! uso scratctl che permette sottotitolo

%Per le Figure
\usepackage[english]{babel}
\usepackage{graphicx}

%simboli matematici strani quali unione disgiunta
\usepackage{amssymb}

%Scrivere Sotto i simboli
\usepackage{amsmath}

%Diagrammi Commutativi
\usepackage{tikz}
\usetikzlibrary{matrix}

%Il simbolo di Identità
\usepackage{dsfont}

%Per riflettere i simboli...
\usepackage{graphicx}


%link iNTERNET
\usepackage{hyperref}

%Enumerate with letters
\usepackage{enumerate}

%Slash over letter
\usepackage{cancel}

%Usare bibiliografia bibtex
%\bibliographystyle{plain}

%Danger sign
\usepackage{fourier}

%:=
\usepackage{mathtools}



%Common symbols
%Common math symbols
	%Number fields
		\newcommand{\Real}{\mathbb{R}}
		\newcommand{\Natural}{\mathbb{N}}
		\newcommand{\Relative}{\mathbb{Z}}
		\newcommand{\Rational}{\mathbb{Q}}
		\newcommand{\Complex}{\mathbb{C}}
	
%equality lingo
	%must be equal
		\newcommand{\mbeq}{\overset{!}{=}} 

% function
	%Domain
		\newcommand{\dom}{\mathrm{dom}}
	%Range
		\newcommand{\ran}{\mathrm{ran}}
	

% Set Theory
	% Power set (insieme delle parti
		\newcommand{\PowerSet}{\mathcal{P}}

%Differential Geometry
	% Atlas
		\newcommand{\Atlas}{\mathcal{A}}
	%support
		\newcommand{\supp}{\textrm{supp}}

	
	
%Category Theory
	%Mor set http://ncatlab.org/nlab/show/morphism
%		\newcommand{\hom}{\textrm{hom}}

%Geometric Lagrangian Mechanics
	% Kinematic Configurations
		\newcommand{\Conf}{\mathtt{C}}
	%Solutions Space
		\newcommand{\Sol}{\mathtt{Sol}}
	%Lagrangian class
		\newcommand{\Lag}{\mathsf{Lag}}
	%Lagrangiana
		\newcommand{\Lagrangian}{\mathcal{L}}
	%Data
		\newcommand{\Data}{\mathsf{Data}}
	%unique solution map
		\newcommand{\SolMap}{\mathbf{s}}
		
		
		
%Peierls (per non sbagliare più)
		\newcommand{\Pei}{Peierls}


%Temporaneo, Aggiunta della mia classe teorem... Deve diventare un pacchetto!
\input{../Latex-Theorem/TheoremTemplateToninus.tex}




\begin{document}

%  Titolo
	\title{AQFT mathematical preliminaries}
	\author{Tony}
	\date{\today}
\maketitle

%  Indice
\tableofcontents

\begin{abstract}
		(sono ripetizioni inutili per la tesi, sono informazioni che si ritrovano ovunque... sono informazioni adatte al "knowledge base")
		
		Fin qui ho basato tutto sugli articoli del mio relatore: \cite{primer}\cite{advances} 
		Molte fonti estendono questi temi dal punto di vista matematico, e.g \cite{bar}
\end{abstract}

%%%%%%%%%%%%%%%%%%%%%%%%%%%%%%%%%%%%%%%%%%%%%%%%%%%%%%%%%%%%%%%%%%%%%%%%%%%%%%%%%%%%%%%%%%%%%%%%%%%%%%%%%%%%%%%%%%%%%%%%%%%%%%%%%%%%%%
\newpage
%\/\/\/\/\/\/\/\/\/\/\/\/\/\/\/\/\/\/\/\/\/\/\/\/\/\/\/\/\/\/\/\/\/\/\/\/\/\/\/\/\/\/\/\/\/\/\/\/\/\/\/\/\/\/\/\/\/\/\/\/\/\/\/\/\/\/\/\/\/\/\/\/\
\section{Globally Hyperbolic SpaceTimes}
%\/\/\/\/\/\/\/\/\/\/\/\/\/\/\/\/\/\/\/\/\/\/\/\/\/\/\/\/\/\/\/\/\/\/\/\/\/\/\/\/\/\/\/\/\/\/\/\/\/\/\/\/\/\/\/\/\/\/\/\/\/\/\/\/\/\/\/\/\/\/\/\/\
	Recurring definitions in general Relativity (excluding the general smooth manifold prolegomena).

	\begin{definition}[SpaceTime]
		A quadruple $(M, g, \mathfrak{o}, \mathfrak{t})$ such that:
		\begin{itemize}
			\item $(M,g)$ is a time-orientable n-dimensional manifold $(n>2)$
			\item $\mathfrak{o}$ is a choice of orientation
			\item $\mathfrak{t}$ is a choice of time-orientation
		\end{itemize}
	\end{definition}

	\begin{definition}[Lorentzian Manifold]
		A pair $(M, g)$ such that:
		\begin{itemize}
			\item $M$ is a n-dimensional $(n\geq2)$, Hausdorff, second countable, connected, orientable smooth manifold.
			\item $g$ is a Lorentzian metric.
		\end{itemize}
	\end{definition}
			
	\begin{definition}[Metric]
		A function on the bundle product of $TM$ with itself: $$g: TM \times_M TM \rightarrow \Real$$ such that the restriction on each fiber $$g_p: T_pM \times T_pM \rightarrow \Real $$ is a non-degenerate bilinear form.
	\end{definition}
	
	\begin{notationfix}
		A Pseudo-riemmanian manifold $(M,g)$ is called:
		 \begin{itemize}
		 	\item \emph{Riemmanian} if the sign of $g$ is positive definite.%, \emph{Pseudo-Riemman} otherwise.
		 	\item \emph{Lorentzian} if the signature is $(+, -, \ldots,- )$ or equivalently $(-,+,\ldots,+)$.
		 \end{itemize}
	\end{notationfix}

	\begin{observation}[Causal Structure]
		If a smooth manifold is endowed with a Lorentzian metric of signature $(+, -, \ldots, -)$ then the tangent vectors at each point in the manifold can be classed into three different types. 
		\begin{notationfix}
			$\forall p \in M, \quad \forall X \in T_pM$, the vector is:
			\begin{itemize}
				\item \emph{time-like} if $g(X,X)>0$.
				\item \emph{light-like} if $g(X,X)=0$.
				\item \emph{space-like} if $g(X,X)<0$.
			\end{itemize}
		\end{notationfix}
	\end{observation}

	\begin{observation}[Local Time Orientability]
		$\forall p\in M$ the timelike tangent vectors in $p$ can be divided into two equivalence classes taking
		\begin{displaymath}
			X \sim Y \; \textrm{iff} \; g(X,Y)>0 \qquad \forall X,Y \in T^\textrm{time-like}_pM:
		\end{displaymath}
		We can (arbitrarily) call one of these equivalence classes "future-directed" and call the other "past-directed". Physically this designation of the two classes of future- and past-directed timelike vectors corresponds to a choice of an arrow of time at the point. 
		\\
		The future- and past-directed designations can be extended to null vectors at a point by continuity.
	\end{observation}
	
	\begin{definition}[Time-orientation]
		A global tangent vector field  $\mathfrak{t}\in \Gamma^\infty(TM)$ over the Lorenzian manifold $M$ such that:
		\begin{itemize}
			\item $\supp(\mathfrak{t}) = M$
			\item $\mathfrak{t}(p)$ is time-like $\forall p \in M$.
		\end{itemize}
	\end{definition}
	\begin{observation}
		The fixing of a time-orientation is equivalent to a consistent smooth choice of a local time-direction.
	\end{observation}	
	
	\begin{definition}[Time-Orientable Lorentzian Manifold]
		A Lorentzian Manifold $(M,g)$ such that exist at least one time-orientation $\mathfrak{t}\in \Gamma^\infty(TM)$.
	\end{definition}

	\begin{notationfix}
		Consider a piece-wise smooth curve $\gamma: \Real\supset I \rightarrow M$ is called:
		\begin{itemize}
			\item \emph{time-like} (resp. light-like, space-like) iff $\dot{\gamma}(p)$ is time-like (resp. light-like, space-like) $\forall p \in M$.
			\item \emph{causal} iff $\dot{\gamma}(p)$ is nowhere spacelike.
			\item \emph{future directed} (resp. past directed) iff is causal and  $\dot{\gamma}(p)$ is future (resp. past) directed $\forall p \in M$.
		\end{itemize}
	\end{notationfix}

	\begin{definition}[Chronological $\substack{\textrm{ future}\\ \textrm{past } }$ of a point]
		Are two subset related to the generic point $p	\in M$:
		\begin{displaymath}
			\mathbf{I}_M^\pm(p) \coloneqq \big\{ q \in M \big\vert \; \exists \gamma \in C^\infty\big((0,1), M\big)\;  \textrm{\footnotesize time-like } \substack{\textrm{future}\\ \textrm{past} } -\textrm{\footnotesize directed }:\; \gamma(0)=p,\; \gamma(1)=q  \big\}
		\end{displaymath}
	\end{definition}
	
	\begin{definition}[Causal $\substack{\textrm{ future}\\ \textrm{past } } $ of a point]
		Are two subset related to the generic point $p	\in M$:
		\begin{displaymath}
			\mathbf{J}_M^\pm(p) \coloneqq \big\{ q \in M \big\vert \; \exists \gamma \in C^\infty\big((0,1), M\big)\; \textrm{\footnotesize causal } \substack{\textrm{future}\\ \textrm{past} } -\textrm{\footnotesize directed }:\; \gamma(0)=p,\; \gamma(1)=q  \big\}
		\end{displaymath}		
	\end{definition}

	\begin{notationfix}
		Former concept can be naturally extended to subset $A \subset M$:
			\begin{itemize}
				\item $\mathbf{I}_M^\pm(A) = \bigcup_{p\in A} \mathbf{I}_M^\pm(p) $
				\item $\mathbf{J}_M^\pm(A) = \bigcup_{p\in A} \mathbf{J}_M^\pm(p) $
			\end{itemize}
	\end{notationfix}

	\begin{definition}[Achronal Set]
		Subset $\Sigma \subset M$ such that every inextensible timelike curve intersect $\Sigma$ at most once.
	\end{definition}

	\begin{definition}[$\substack{\textrm{ future}\\ \textrm{past } } $ Domain of dependence of an Achronal set]
		The two subset related to the generic achornal set $\Sigma \subset M$:
		\begin{displaymath}		
			\mathbf{D}_M^\pm(\Sigma) \coloneqq \big\{ q \in M \big\vert \; \forall \gamma \substack{\textrm{ past}\\ \textrm{ future} }\textrm{\footnotesize inextensible causal curve passing through }q : \; \gamma(I) \cap \Sigma \neq \emptyset  \big\}
		\end{displaymath}		
	\end{definition}

	\begin{notationfix}
		$\mathbf{D}_M(\Sigma)  \coloneqq \mathbf{D}_M^+(\Sigma) \cup \mathbf{D}_M^-(\Sigma)$ is called \emph{total domain of dependence}.
	\end{notationfix}

	\begin{definition}[Cauchy Surface]
		Is a subset $\Sigma \subset M$ such that:
		\begin{itemize}
			\item closed
			\item achronal
			\item $\mathbf{D}_M(\Sigma) \equiv M$
		\end{itemize}
	\end{definition}

	The term "hypersurface" is not used by chance:	
	\begin{proposition}
		Every Cauchy surface $\Sigma$ is a three dimensional, embedded, $C^0$ submanifold of $M$
	\end{proposition}
	\begin{proof}
		See Wald (general relativity) teo 8.3.1
	\end{proof}
	

		\begin{Warning}
		\danger copiato da \cite{primer}
		From a physical point of view, we are interested in those spacetimes which allow to set a well-posed
initial value problem for hyperbolic partial differential equations, such as the scalar D’Alambert wave equation, to quote the simplest, yet most important example. In particular we need to ensure that the spacetime we consider possesses at least one distinguished codimension 1 hypersurface on which we can assign the initial data needed to construct a solution of such an equation.	
		\end{Warning}	
		
		\begin{definition}[Globally-Hyperbolic SpaceTime]\label{Def:GHSP}
			 Spacetime $M$ such that there exists at least one \emph{Cauchy Surface}
		\end{definition}
		
		According to Definition \ref{Def:GHSP}, only the existence of a single Cauchy hypersurface is guaranteed. 
		This is slightly disturbing since there is no reason a priori why an initial value hypersurface for a certain partial differential equation should be distinguished. 
		This quandary has been overcome proving that, if a spacetime $(M,g)$ is globally hyperbolic, then there exists a foliation of $M$ by Cauchy surfaces:
		
			\begin{theorem}[Globally hyperbolic  space characterization]\label{Teo:GHSC_character}
				Let $(M,g)$ be any time-oriented spacetime. The following two statements are equivalent:
				\begin{itemize}
					\item $(M,g)$ is globally hyperbolic.
					\item $(M,g)$ is isometric to $ \Real \times \Sigma $ 
						endowed with the line element $ds^2 = \beta dt^2 - h_t$ 
						where $t : \Real \times \sigma \rightarrow \Real$ is the projection on the first factor, 
						$\beta$ is a smooth and strictly positive function on $\Real \times \Sigma$ 
						and $t \mapsto h_t , t \in \Real$, yields a one-parameter family of smooth Riemmanian metrics.\\
						Furthermore, for all $t\in \Real$, $\{t\}\times \Sigma$ is an (n−1)-dimensional, spacelike, smooth Cauchy surface in M.
				\end{itemize}
			\end{theorem}
			\begin{proof}
				Teo 3.17 in J. K. Beem, P. E. Ehrlich and K. L. Easley, Global Lorentzian Geometry.
				\\
				section 1.3 in \cite{barwav}
			\end{proof}

		To conclude this section, we introduce some terms which will be often used in the following in order to
specify the support properties of the sections of a vector bundle with base a globally hyperbolic spacetime.
		\begin{notationfix}
		Let $M$ be a globally hyperbolic spacetime and $E=(E,\pi,M;V)$ a vector bundle of typical fiber $V$.
		We denote:
		\begin{itemize}
			\item $\Gamma_0(E)$ the space of \emph{compactly supported} smooth sections.
			\item $\Gamma_{sc}(E)$  the space of \emph{spacelike compact} smooth sections.\\
				$\big(\; f\in \Gamma_{sc}(E)$ if there exists a compact subset $K \subset M$  such that $\supp f \subset \mathbf{J}_M(K)$. $\big)$
			\item  $\Gamma_{fc}(E) $ the space of \emph{future- compact} smooth sections.\\
				$\big(\; f\in \Gamma_{fc}(E) $ if  $\supp(f) \cap  \mathbf{J}^+_M(K)$ is compact for all $p\in M$.$\big)$
			\item  $\Gamma_{pc}(E) $ the space of \emph{past- compact} smooth sections.\\
				$\big(\; f\in \Gamma_{pc}(E) $ if  $\supp(f) \cap  \mathbf{J}^-_M(K)$ is compact for all $p\in M$.$\big)$
			\item $\Gamma_{tc}(E) \coloneqq \Gamma_{pc}(E) \cap \Gamma_{fc}(E) $ the space of \emph{timelike compact} smooth sections.
		\end{itemize}
		\end{notationfix}
		
		This class of manifolds includes most of the physically interesting examples, e.g.: Minkowski spacetimes, Friedman-Robertson-Walker solutions, Kerr family. \cite{advances}
					
			A trivial example:		
			\begin{example}
				Trivially, the real line $\Real$ is a globally hyperbolic manifold.
				\\
				Each point $x\in \Real$ represent a proper Cauchy surfaces which realize the trivial foliation $\Real \simeq 1\times \Real $ required by theorem \ref{Teo:GHSC_character}
			\end{example}
		

\newpage
%\/\/\/\/\/\/\/\/\/\/\/\/\/\/\/\/\/\/\/\/\/\/\/\/\/\/\/\/\/\/\/\/\/\/\/\/\/\/\/\/\/\/\/\/\/\/\/\/\/\/\/\/\/\/\/\/\/\/\/\/\/\/\/\/\/\/\/\/\/\/\/\/\
\section{Linear Differential Operator}
%\/\/\/\/\/\/\/\/\/\/\/\/\/\/\/\/\/\/\/\/\/\/\/\/\/\/\/\/\/\/\/\/\/\/\/\/\/\/\/\/\/\/\/\/\/\/\/\/\/\/\/\/\/\/\/\/\/\/\/\/\/\/\/\/\/\/\/\/\/\/\/\/\
Basic Definition in L.P.D.O. on smooth vector sections.
\\
		\begin{Warning}
		(ADVANCES)\\
		Globally hyperbolic spacetimes play a pivotal role, not only because they do not allow for pathological situations, such as closed causal curves, but also because they are the natural playground for classical and quantum fields on curved backgrounds. 
		More precisely, the dynamics of most (if not all) models, we are interested in, is either ruled by or closely related to wave-like equations. Also motivated by physics, we want to construct the associated space of solutions by solving an initial value problem. 
		To this end we need to be able to select both an hypersurface on which to assign initial data and to identify an evolution direction. In view of Theorem 1, globally hyperbolic spacetimes appear to be indeed a natural choice. 
		Goal of this section will be to summarize the main definitions and the key properties of the class of partial differential equations, useful to discuss the models that we shall introduce in the next sections. 
		Since this is an overkilled topic, we do not wish to make any claim of being complete and we recommend to an interested reader to consult more specialized books and papers for more details.
		\end{Warning}

	\subsection{L.P.D.O}
		Consider $E=(E,\pi,M;V), E'=(E',\pi',M;V')$ two linear vector bundle over $M$ (with different typical fiber), we define:
		\begin{definition}[Linear Partial Differential operator \footnotesize( of order at most $s\in \Natural_0$)]\label{Def:LPDO}
			Linear map $L:\Gamma(E)\rightarrow \Gamma(E')$ such that $\forall p \in M$ exists:
		\begin{itemize}
			%\item open set $U \ni p$.
			%\item $(U, \varphi )$ local chart on $M$.
			%\item $(U, \chi)$ local trivialization of $F$
			%\item $(U, \chi')$ local trivialization of $F'$
			\item $U \ni p$ open set rigged with:
				\begin{itemize}
					\item $(U, \varphi )$ local chart on $M$.
					\item $(U, \chi)$ local trivialization of $F$
					\item $(U, \chi')$ local trivialization of $F'$
				\end{itemize}
			\item $\{A_\alpha :U \rightarrow \Hom(V,V')\; \vert \: \alpha \in \Natural_0^n, \vert \alpha \vert \leq s \}$ collection of smooth maps labeled by multi-indices.
		\end{itemize}
		which allows to express $L$ locally:
		\begin{displaymath}
			\chi' \circ ( L \sigma) \circ \varphi^{-1} =
			\sum_{\vert \alpha \vert \leq s} A_\alpha \partial^\alpha (\chi \circ \sigma \circ \varphi^{-1} ) 
			\qquad \forall \sigma \in \dom(L) \subset \Gamma(E)
		\end{displaymath}
	\end{definition}
	
		\begin{notationfix}
			Using implicitly the coordinate charts \cite{advances}:
			\begin{displaymath}
				L \big(\sigma \big\vert_U\big) = \sum_{\vert \alpha \vert \leq s} A_\alpha \partial^\alpha \sigma \qquad \forall \sigma \in \Gamma(E)
			\end{displaymath}
		\end{notationfix}
		\begin{remark}
			(multi-index notation)
			\\
			A multi-index is a natural valued finite dimensional vector $\alpha = ( \alpha_0, \ldots, \alpha_n-1) \in \Natural^n_0$ with $n<\infty$.
			\\
			On $\Real^n$ a general differential operator can be identified by a multi-index:
			\begin{displaymath}
				\partial^\alpha = \prod_{\mu = 0}^{n-1} \partial_\mu ^{\alpha_\mu}
			\end{displaymath}
			(Until the Schwartz theorem holds, the order of derivation is irrelevant.)
			\\
			The order of the multi-index is defined as:
			\begin{displaymath}
				\vert \alpha \vert \coloneqq \sum_{\mu=0}^{n-1} \alpha_\mu
			\end{displaymath}
		\end{remark}
	
		\begin{observation}
			Notice that linear partial differential operators cannot enlarge the support of a section.
		\end{observation}		
		
		
		\begin{notationfix}
			We say that $L$ is exactly of order $k$ if it is of order at most $k$, but not of order at most $k-1$.
		\end{notationfix}
		\begin{observation}
			Notice that, as a consequence of their definition, linear partial differential operators cannot enlarge the
support of a section.
		\end{observation}
		
		Definition \ref{Def:LPDO} accounts for a large class of operators, most of which are not typically used in the framework of field theory, especially because they cannot be associated to an initial value problem. 
		In order to select a relevant subset we introduce a useful concept ( see also \cite{barwav}[pag 172]).
		
		Consider a Linear differential operator $L: \Gamma(E) \rightarrow \Gamma(E')$:
		\begin{definition}[Principal Symbol]
		 	The map $\sigma_L: T^*M \rightarrow \Hom(E,E') $locally defined as follows: \\
		 	For a given $p\in M$, consider a coordinate chart $(U, x^i)$ around $p$ and local trivializations of $E$ and of $E'$ (as precribed in Definition \ref{Def:LPDO}).
		 	\\ 
		 	For all $\xi = \xi_i dx^i \in T^*_pM$ set:
		 	\begin{displaymath}
		 		\sigma_L \big(\xi \big) = \sum_{|\alpha|=s}  \xi^\alpha A_\alpha (p) 
		 	\end{displaymath}
		 	where $ \xi^\alpha = \prod_{\mu=0}^{m-1} \xi^{\alpha_\mu}$
		 \end{definition}
		
		Remembering the definition of Natural pairing (see \cite{TonyBundles}) we can define a dual operator of a L.P.D.O. $L: \Gamma(E) \rightarrow \Gamma(G)$:
		\begin{definition}[Formal Dual Operator]
			Linear partial differential operator $L^\star: \Gamma(G^*) \rightarrow \Gamma(E^*)$ such that:
			\begin{displaymath}
				<L^\star g' , f > = <g', L f> 
			\end{displaymath}
			$\forall f\in \Gamma(E),\; g' \in \Gamma(G^*)$ which have supports with compact overlap.
		\end{definition}
		\begin{proposition}
			$\forall P $ linear partial differential operator, $\exists1! P^\dagger$.
		\end{proposition}
		
		Similarly , in presence of a bundle inner product $g: E\otimes_M E \rightarrow \Real$ and h respectively on $E$ and $G$, can be defined an adjoint operator through the dual space isomorphism:
		\begin{definition}[Formal Adjoint Operator]
			Linear partial differential operator $L^\dagger : \Gamma(G) \rightarrow \Gamma(E)$ such that:
			\begin{displaymath}
				g(L^\dagger g ,f)  = h(g , L f)
			\end{displaymath}
			$\forall f\in \Gamma(E), g \in \Gamma(G)$ with compact supports intersection.		
		\end{definition}		
		
		\begin{notationfix}
			In presence of an inner product $g$ , denote $f^*=g(f, \cdot)	\in \Gamma^*(E)$ the dual section associated to $f \in \Gamma(E)$ through $g$.
			Obviously:
			\begin{displaymath}
				<f^*, f'> \equiv g(f,f')
			\end{displaymath}
		\end{notationfix}
		
		\begin{notationfix}
			$L:\Gamma(E) \rightarrow \Gamma(E)$ is \emph{self-adjoint} whenever $L^\dagger = L$.
		\end{notationfix}
				
	\subsection{Green Operators}
		\begin{NB}
			From now on we will consider only bundles with globally-hyperbolic spacetime base.
		\end{NB}	
		Let $M$ be a globally hyperbolic spacetime, consider a vector bundle $E$ over $M$ and a L.p.d.o. $L: \Gamma(E) \rightarrow \Gamma(E)$:
		\begin{definition}[$\substack{\textrm{ retarded}\\ \textrm{advanced } } (\pm)$ Green Operators]
			L.p.d.o. $G^\pm : \Gamma (E) \rightarrow \Gamma(E)$ such that:
			\begin{itemize}
			\item $\dom(G^+) = \Gamma_{pc}(E) \qquad \dom(G^-) = \gamma_{fc}(E)$
			\item $LG^\pm f=G^\pm Lf = f \qquad \forall f\in \dom(G^\pm)$
			\item $\supp(G^\pm f) \subset \mathbf{J}^\pm_M (\supp(f)) \qquad \forall f\in \dom(G^\pm)$
			\end{itemize}
		\end{definition}
		\begin{definition}[ Advanced $(-)$ Green Operator of $L$]
			L.p.d.o. $G^- : \Gamma_{fc} (E) \rightarrow \Gamma(E)$ such that:
			\begin{itemize}
			\item $\dom(G^-) = \Gamma_{fc}(E)$
			\item $LG^-f=G^-Lf = f \qquad \forall f\in \Gamma_{fc}(E)$
			\item $\supp(G^- f) \subset \mathbf{J}^-_M (\supp(f)) \qquad \forall f\in \Gamma_{fc}(E)$
			\end{itemize}
		\end{definition}
		\begin{definition}[ Retarded $(+)$ Green Operator of $L$]
			L.p.d.o. $G^+ : \Gamma_{pc} (E) \rightarrow \Gamma(E)$ such that:
			\begin{itemize}
			\item $\dom(G^+) = \Gamma_{pc}(E)$
			\item $LG^+f=G^+Lf = f \qquad \forall f\in \Gamma_{pc}(E)$
			\item $\supp(G^+ f) \subset \mathbf{J}^+_M (\supp(f)) \qquad \forall f\in \Gamma_{pc}(E)$
			\end{itemize}
		\end{definition}
		\begin{observation}
			From the definition follows that $G^\pm$ is the left-right inverse of the restriction of $L$ to $\dom(G^\pm)$.
		\end{observation}


		
		\begin{notationfix}
			We refer to the operator:
			\begin{displaymath}
				E \coloneqq G^-  - G^+ : \Gamma_{tc}(E) \rightarrow \Gamma(E)
			\end{displaymath}
			as the \emph{Advanced minus Retarded operator} or \emph{Causal Propagator}\cite{primer}.
		\end{notationfix}		
		
		Green operators are not necessarily unique. For this we introduce the following definition:
		\begin{definition}[Green hyperbolic operator]
			The linear partial differential operator $P$ is be called Green hyperbolic if $P$ and $P^\star$ have advanced and retarded Green’s operators.
		\end{definition}
		for such operators uniqueness of Green's operator comes from free:
		\begin{theorem}[Characterization of Green Hyperbolic operators]
			\begin{hypothesis}
				\item $E=(E,\pi,M)$ a vector bundles over a globally hyperbolic spacetime $M$.
				\item $P:\Gamma(E) \rightarrow \Gamma(E)$ a green hyperbolic operator, $G^\pm$ its Green's operators and $G_\star^\pm$ the Green's operators of the dual.
			\end{hypothesis}		
			\begin{thesis}
				\item $L$ posses an unique retarded $\GreenRet$ and advanced $\GreenAdv$ green operator.
				\item $<G_\star^\pm f', f> = <f', G^\mp f > \qquad \forall f \in \Gamma_0(E),\: \forall f' \in \Gamma_0(E^*)$
			\end{thesis}
		\end{theorem}
		\begin{proof}
			see \cite{advances}[proposition 2]
		\end{proof}
		
		The same result can be obtained considering the pairing induced by an inner product:
		\begin{corollary}
			see \cite{advances}[Lemma 1]
		\end{corollary}
		From that follows:
		\begin{corollary}
			If $L$ is Green hyperbolic and self-adjoint, then:
			\begin{itemize}
				\item $\exists 1 ! \; G^\pm$ Green's operators of $L$
				\item $(G^\pm)^\dagger = G^\mp$
			\end{itemize}
		\end{corollary}

	\subsection{Cauchy problem and Solution Space}
		The globally-hyperbolic condition property of the base manifold $M$ is what allows to define \emph{Cauchy problems} associated to a linear partial differential operator $P:\Gamma(E) \rightarrow \Gamma(E)$:
		\begin{remark}
			For every cauchy surface $\Sigma \subset M$, for each couple $(u_0, u_1) \in \Gamma(\Sigma) \times \Gamma(\Sigma)$ we can state a \emph{Cauchy Problem}:
			\begin{equation}\label{CauchyProblem}
				\begin{cases} P u = 0 \\ u = u_0 \\ \nabla_{\vec{n}}u= u_1 \end{cases}
			\end{equation}
			(Someties we refer to $P u = 0$ as \emph{Wave equation} of $u\in \Gamma(E)$.)
		\end{remark}
		\begin{definition}[PDE-hyperbolic operator]
			L.d.p.o. $P$ such that exists an unique solution for every Cauchy problem \ref{CauchyProblem}
		\end{definition}

	\begin{observation}
	"Green-hyperbolic operators are not necessarily hyperbolic in any PDE-sense and that they cannot be characterized in general by well-posedness of a Cauchy problem.	" \cite{Terlaky2010} \cite{Bar2010}
	\\
	However the existence and uniqueness can be proved for the large class of the \emph{Normally-Hyperbolic Operators}.
	\end{observation}

	\subsection{Normally Hyperbolic Operators}
		Is a class of L:P.D.O. hyperbolic in both PDE and Green sense.
		Given a Lorentzian manifold $(M,g)$ and two vector bundles $E=(E,\pi,M;V), E'=(E',\pi',M;V')$,
		\begin{definition}[Normally Hyperbolic Operators]
			Second order linear partial differential operator $P:\Gamma(E)\rightarrow \Gamma(E')$ such that:
			\begin{displaymath}
				\sigma_P(\xi) = g(\xi,\xi) \IdOp_{E_p} \qquad \forall p\in M, \xi \in T^*_pM
			\end{displaymath}
		\end{definition}
		
		\begin{observation}%In coordinate questi operatori hanno un'espressione molto familiare ...
			Making explicit the coordinate expression of a normally hyperbolic operator $P$ , one realizes how such operators  provide the natural generalization of the usual Wave operator. 
			\\
			Consider a globally hyperbolic operator $P$ for all $p \in M$ a trivializing chart $(U, \varphi, \chi)$ centered in $p$. 
			There exist a collection $\big\{A, A^\mu \vert \mu\in \{0, \ldots ,m-1\}\big\}$ of smooth 
			$\Hom(V,V)-$valued maps on $U$ such that, $P$ reas as follows:
			\begin{displaymath}
				\chi \circ ( P \sigma) \circ \varphi^{-1} =
				\big( g^{\mu \nu} \IdMap_V \partial\mu \partial_nu + A^\mu \partial_\mu + A\big)
				(\chi \circ \sigma \circ \varphi^{-1} ) 
				\qquad \forall \sigma \in \dom(P) \subset \Gamma(E)
			\end{displaymath}
		where both the chart and the vector bundle trivializations are understood. 
		One immediately notices that locally this expression agrees up to terms of lower order in the derivatives with the one for the d'Alembert operator acting on sections of $E$  constructed out of a covariant derivative $\nabla$ on $E$, that is the operator:
		\begin{displaymath}
			\square_\nabla = g^{\mu \nu} \nabla_\mu \nabla_\nu : \Gamma(E) \rightarrow \Gamma(E)
		\end{displaymath}
		\end{observation}
		This definition becomes even more important if we assume, moreover, that the underlying background is globally hyperbolic, since we can associate to each normally hyperbolic operator P an initial value problem and talk about Green's operator.
		\begin{proposition}[Green operators]
			Be $P$ normally hyperbolic operator, then:
			\begin{itemize}
				\item	$P^\star$ is a normally hyperbolic operator.
				\item $P$ is Green hyperbolic.
			\end{itemize}
		\end{proposition}	
		\begin{proof}
			\cite{barwav}[Corollary 3.4.3]
		\end{proof}
		
	\begin{proposition}[Existence and uniqueness for the Cauchy Problem]
	$ $
		\begin{hypothesis}
			\begin{itemize}
				\item $E=(E,\pi,M;V)$ a vector bundle on $M = (M,g,\mathfrak{o},\mathfrak{t}) $, globally hyperbolic spacetime.
				\item $\Sigma \subset M$ a spacelike Cauchy surface with future-pointing unit normal vector field $\vec{n}$.
				\item $P$ a normally hyperbolic operator and $\nabla$ a P-compatible\footnote{There existss a section $A \in \Gamma(\textrm{End}(E))$ such that $\square_\nabla + A = P$.} covariant derivative on $E$
			\end{itemize}
		\end{hypothesis}
	\begin{thesis}
		\begin{itemize}
			\item The Cauchy problem;
				\begin{displaymath}
					\begin{cases} 
						P u = J & \textrm{on $M$} \\ 
						u = u_0 & \textrm{on $\Sigma$}\\ 
						\nabla_{\vec{n}}u= u_1  & \textrm{on $\Sigma$}
					\end{cases}
				\end{displaymath}
				admit a unique solution $u\in \Gamma(E)$ for any $J\in \Gamma(E)$ and $ u_0,u_1 \in \Gamma(\Sigma)$
			\item $\supp(U) \subset \mathbf{J}_M \big( \supp(u_0) \cup \supp(u_0) \cup \supp(J) \big)$
		\end{itemize}
	\end{thesis}
	\end{proposition}	
	\begin{proof}
		The proof of this proposition has been given in different forms in several books, \cite{barwav}6, 26] and in [3, Corollary 5]. Notice that equation (2) is not linear since we
	\end{proof}
		
	\subsection{Continuity}
		(...) see  p.15-16 \cite{primer}


\newpage
%\/\/\/\/\/\/\/\/\/\/\/\/\/\/\/\/\/\/\/\/\/\/\/\/\/\/\/\/\/\/\/\/\/\/\/\/\/\/\/\/\/\/\/\/\/\/\/\/\/\/\/\/\/\/\/\/\/\/\/\/\/\/\/\/\/\/\/\/\/\/\/\/\
\section{*-Algebras}
%\/\/\/\/\/\/\/\/\/\/\/\/\/\/\/\/\/\/\/\/\/\/\/\/\/\/\/\/\/\/\/\/\/\/\/\/\/\/\/\/\/\/\/\/\/\/\/\/\/\/\/\/\/\/\/\/\/\/\/\/\/\/\/\/\/\/\/\/\/\/\/\/\

			\begin{definition}[Algebra over a field]
				A pair $\big( (V,\KField), \cdot )$ which consists of a vector space over field $\KField$ and a multiplication operator $\cdot: V \times V \rightarrow V$ such that:
				\begin{enumerate}
					\item is bilinear.
					\item satisfies left/right distributivity :
						\begin{displaymath}
							(x+y) \cdot z = x \cdot z +y \cdot z \qquad x \cdot (y+z) = x \cdot y +x \cdot z \qquad \forall x,y,z \in V
						\end{displaymath}			
					\item is scalar compatible:	
						\begin{displaymath}
							(\alpha x) \cdot (\beta y) = (\alpha \beta) x \cdot y \qquad \forall \alpha,\beta \in \KField \; \forall x,y \in V
						\end{displaymath}						 
				\end{enumerate}		
			\end{definition}		

			\begin{definition}[Unital Algebra (over a field)]
				Algebra $(V,\cdot)$ over field $\KField$ such that:
				\begin{enumerate}
					\item $\exists \IdOp \in V$ such that $\IdOp \cdot u = u \cdot \IdOp = u \qquad \forall u \in V$
					\item associativity:
						\begin{displaymath}
							v \cdot ( w \cdot u) = (v \cdot w)\cdot u = v \cdot w \cdot u \qquad \forall u,v,w \in V
						\end{displaymath}
				\end{enumerate}
			\end{definition}		

			\begin{definition}[Algebra (over a field) generate by a subspace $W\subset V$]
				Is an Algebra $(V,\cdot)$ over field $\KField$ such that each elements of $V$ can be obtained as a polynomial in the elements of $W$:
				\begin{displaymath}
					V = \textrm{span}\big\{\prod_i W_i \big\vert \{w_i\} \subset W \big\}
				\end{displaymath}
			\end{definition}		
			
			\begin{definition}[*-algebra over a field]
				Triple $\big( (V,\KField), \cdot, *\big)$ where $\big((V,\KField), \cdot\big)$ constitutes an unital algebra over field $\KField$ and $*:V\rightarrow V$ is an \emph{involutive} map :
				\begin{enumerate}
					\item $ (\alpha x + y)^* = \bar{\alpha} x^* + y^* \qquad \forall \alpha \in \KField, \; \forall x,y \in V$
					\item $(x \cdot y)^* = y^* \cdot x^* \forall x,y \in V$
					\item $\IdOp^* = \IdOp$
					\item $ (x^*)^* = x \qquad \forall x \in V$
				\end{enumerate}	
			\end{definition}
			
			
			\begin{observation}
			 An algebra over a field $\KField$ is a ring\cite{TonyAlgebra} $A$ together with a ring homomorphism
			\begin{displaymath}
				\eta : \KField \rightarrow Z(A)
			\end{displaymath}
			where $Z(A)$ is the center of $A$ \url{https://en.wikipedia.org/wiki/Algebra_over_a_field#Algebras_and_rings}.
			
				\begin{remark}
				The center of a ring $R$ is the subset of $R$ consisting of all those elements $x \in R$ such that 
				\begin{displaymath}
					x r = r x \qquad \forall x \in R
				\end{displaymath}
				The center is a commutative subring of $R$, and $R$ is an algebra over its center.\\
				The center of an algebra $A$ consists of all those elements $x \in A$ such that 
				\begin{displaymath}
					x a = a x \qquad \forall a \in A
				\end{displaymath}		
				\end{remark}				 
			\end{observation}

			\begin{remark}
				Let $R= (R, \otimes, \oplus)$ a ring, $ I\subseteq R$ is a \emph{bilateral ideal} if:
				\begin{itemize}
					\item 	$(I,\oplus) \subseteq ( R, \oplus)$ is a (additive) subgroup.
					\item \emph{Right Ideal}
						\begin{displaymath}
							x \otimes r \in I \qquad \forall x \in I , \: r \in R
						\end{displaymath}									
					\item \emph{Left Ideal}
						\begin{displaymath}
							r \otimes x \in I \qquad \forall x \in I , \: r \in R
						\end{displaymath}									
				\end{itemize}
				(Obviously $I$ is a subring.)
			\end{remark}
			
			\begin{definition}[(bilateral) Ideal]
				Let $A= ((A,\KField), \otimes, \oplus)$ an unital algebra over $\KField$, $ I\subseteq R$ is a \emph{bilateral ideal} if:
				\begin{itemize}
					\item 	$((I, \KField),\oplus) \subseteq ( (R, \KField), \oplus)$ is a subvector space.
					\item \emph{Right Ideal}
						\begin{displaymath}
							x \otimes r \in I \qquad \forall x \in I , \: r \in R
						\end{displaymath}									
					\item \emph{Left Ideal}
						\begin{displaymath}
							r \otimes x \in I \qquad \forall x \in I , \: r \in R
						\end{displaymath}									
				\end{itemize}
				(Obviously $I$ is a subalgebra.)				
			\end{definition}
			
			\begin{remark}
				Let be $(R, \oplus)$ a ring and $I$ a bilateral ideal.\\
				We call \emph{quotient ring (of $R$ modulo $I$)} the set:
				\begin{displaymath}
					\frac{R}{I}= \big\{[a] \equiv a + I \coloneqq \{a+r \: \vert \: r \in I\}\quad \big\vert\quad  a \in R\big\}
				\end{displaymath}
				constituted by the equivalence classes $[a]$ with respect to the congruence relation:
				\begin{displaymath}
					a \sim b \quad \Leftrightarrow (a \ominus b) \in I
				\end{displaymath}
				and rigged with the ring operations:
				\begin{displaymath}
					[a]+[b] = [ a \oplus b] \qquad \forall a,b \in R
				\end{displaymath}
				\begin{displaymath}
					[a]\cdot[b] = [ a \otimes b] \qquad \forall a,b \in R
				\end{displaymath}
			\end{remark}
			
			\begin{definition}[Quotient Algebra of $A$ modulo $I$]
				
			\end{definition}


		
%-_-_-_-_-_-_-_-_-_-_-_-_-_-_-_-_-_-_-_-_-_-_-_-_-_-_-_-_-_-_-_-_-_-_-_-_-_-_-_-_-_-_-_-_-_-_-_-_-_-_-_-_-_-_-_
%-_-_-_-_-_-_-_-_-_-_-_-_-_-_-_-_-_-_-_-_-_-_-_-_-_-_-_-_-_-_-_-_-_-_-_-_-_-_-_-_-_-_-_-_-_-_-_-_-_-_-_-_-_-_-_
\newpage
\begin{thebibliography}{100}

\bibitem{advances} Benini, M. and Dappiaggi, C. in Advances in AQFT 1–49

\bibitem{primer} Benini, M., Dappiaggi, C. and Hack, T.-P. Quantum Field Theory on Curved Backgrounds — a Primer. Int. J. Mod. Phys. A 28, 1330023 (2013).

\bibitem{barwav} Bar, C., Ginoux, N. and Pfaeffle, F. Wave Equations on Lorentzian Manifolds and Quantization. (2008).

\bibitem{bar}  Bar, C. Green-hyperbolic operators on globally hyperbolic spacetimes. 1–26 (2010).

\bibitem{barconcepts}

\bibitem{TonyBundles}  Toninus, P. An Excursus on Bundles. at \url{https://github.com/MasterToninus/Dispensarium/blob/master/Fiber Bundles/FiberBundles.pdf}

\bibitem{TonyAlgebra} Toninus, P. Abstact Algebra Zoology. at \url{https://github.com/MasterToninus/Dispensarium/blob/master/Remarks in Algebra/RemarkAlgebra.pdf}

\end{thebibliography}		


\end{document}