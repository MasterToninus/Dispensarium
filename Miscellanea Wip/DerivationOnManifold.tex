%	Appunto di carralleta sui concetti di derivazione in geometria differenziale!
%
%


\documentclass[a4paper,12pt]{scrartcl}    %attenzione! uso scratctl che permette sottotitolo

% PREAMBLE %%%%%%%%%%%%%%%%%%%
\usepackage{hyperref}






%%%%%%%%%%%%%%%%%%%


\begin{document}

%  Titolo
	\title{MEMENTO:}
	\subtitle{Derivazione in Varieta' differenziabili}
	\author{Tony}
	\date{\today}
\maketitle

%introduzione
bla bla

%  Indice
\tableofcontents

\section{Derivata in Astratto}
\url{http://en.wikipedia.org/wiki/Derivation_%28differential_algebra%29]}

\section{Derivata di Lie}
\href{run:d:~/Downloads/FoM2.pdf}{link}
Jurgen Jost, Riemannian Geometry and Geometric Analysis, (2002) Springer-Verlag, Berlin ISBN 3-540-42627-2 See section 1.6.


\section{Derivata esterna}
Capitolo 2.4 di Abraham Marsden

\section{Derivata Covariante}
Capitolo 2.7 di Abraham Marsden

\section{Vettori tangenti come derivazione}
Capitolo 2.2 di Abraham Marsden



\end{document}