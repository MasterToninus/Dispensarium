\documentclass[a4paper,10pt]{scrartcl}
\usepackage[utf8]{inputenc}

\usepackage[tensors]{../Math-Symbols-List/toninus-math-symbols}





%opening
\title{CheatSheet Tensor Algebra}
\author{Tony}

\begin{document}

\maketitle

%\begin{abstract}
%\end{abstract}

\section{Factoid}\label{factoid}
(TODO: da capire. tutto ciò valeanche se V è solo un modulo su una anello invece che spazio vettoriale su campo. Inoltre anche la proprietà di algebra è conservata)

The space of all the function valued in a vector space are a vector space itself.
\\
Namely, for any set $\Omega$ and vector space $(V,\KField)$ the set 
\begin{displaymath}
 V^\Omega = \{ x : \Omega \rightarrow V \st x(i) = x_i \} 
\end{displaymath}
endowed with the functions:
\begin{displaymath}
 +: V^\Omega \times V^\Omega \rightarrow V^\Omega \quad s.t. (x+y)_i = x_i +y_i \quad \forall i \in \Omega;\: \forall x,y \in V^\Omega;
\end{displaymath}
\begin{displaymath}
 \cdot : \KField	\times V^\Omega \rightarrow V^\Omega \quad s.t. (\lambda x)_i = \lambda x_i \quad \forall i \in \Omega;\: \forall x \in V^\Omega; \: \forall \lambda \in \KField
\end{displaymath}

e.g.:
\begin{displaymath}
 \Real^\Omega = \Real^n \quad \textrm{with } \Omega= \{i \in \Natural \st 1 \leq i \leq n \}
\end{displaymath}
Come dicevo prima ho che le funzioni a valori in R formano un algebra su campo R.

\section{Tensor (Linear) Algebra}
Let's see at this stage the \emph{Tensor Calculus} as a machinery able to build up a plethora of other vector spaces starting from a single prototype.\\
The main blocks to accomplish this construction are two.

Let be $V$, $W$ two vector space on the same field $\KField$

\begin{displaymath}
 V^\ast = \{ \alpha : V \rightarrow \KField \st \alpha \textrm{linear} \} \subset \KField^V
\end{displaymath}

\begin{displaymath}
 V \otimes_\KField W = \{ \pi: V^\ast \times W^\ast \rightarrow \KField \st \pi \textrm{bilinear} \} \subset \KField ^{(V^\ast \times W^\ast)}
\end{displaymath}

Other than the tensor product of space is possible to multiplicate covector.

(warning sloppy abused notation
$$Def---$$
\begin{displaymath}
 \otimes : V^\ast \times W^\ast \rightarrow V \otimes_\KField W  \st (\alpha \otimes \beta ) ( v,w) = \alpha(v) \beta(w)
\end{displaymath}





The last inlcusion relations are there to remind us that they are naturally vector spaces according to \ref{factoid}.

From there we define:
\begin{displaymath}
 \TensorSpace^p (V) = \TensorSpace^p_0(V) = \{ \pi: \underbrace{V^\ast \times \ldots \times V^\ast}_\text{p times} \rightarrow \KField \st \textrm{multilinear} \} = \underbrace{V \otimes \ldots \otimes V}_\text{p times}
\end{displaymath}
\begin{displaymath}
 \TensorSpace_q (V) = \TensorSpace^0_q(V) = \{ \pi: \underbrace{V \times \ldots \times V}_\text{q times} \rightarrow \KField \st \textrm{multilinear} \} = \underbrace{V^\ast \otimes \ldots \otimes V^\ast}_\text{q times}
\end{displaymath}
\begin{displaymath}
 \TensorSpace^p_q (V) = \TensorSpace^p(V) \otimes \TensorSpace_q(V) = \{ \pi: \underbrace{V \times \ldots \times V}_\text{q times} \times \underbrace{V^\ast \times \ldots \times V^\ast}_\text{p times} \rightarrow \KField \st \textrm{multilinear} \}
\end{displaymath}
\begin{displaymath}
 \TensorSpace^0_0(V) = \KField
\end{displaymath}
\begin{displaymath}
 \TensorSpace^\bullet(V) = \bigoplus_{p \geq 0} \TensorSpace^p(V)
\end{displaymath}
\begin{displaymath}
 \TensorSpace_\bullet(V) = \bigoplus_{q \geq 0} \TensorSpace_q(V)
\end{displaymath}
\begin{displaymath}
 \TensorSpace(V) = \bigoplus_{p,q \geq 0} \TensorSpace^p_q(V)
\end{displaymath}

The definition of tensor product of forms can be easily extend to Tensors. Therefore:\\
$\left( \TensorSpace(V), \otimes \right)$ forms a \emph{Associative Algebra over $\KField$}.


\section{Exterior Algebra}
\begin{displaymath}
 V \wedge_\KField V = \left\{ \alpha: V^\ast \times V^\ast \rightarrow \KField \st \textrm{multilinear, skewsymmetric} \right\}
\end{displaymath}

$$--- questa e^- giusta ---$$
\begin{displaymath}
 \wedge : V^\ast \times V^\ast \rightarrow V \wedge_\KField V  \st (\alpha \wedge \beta ) ( v,w) = \alpha(v) \beta(w) -\beta(v)\alpha(w)
\end{displaymath}



$$--- oss ---$$
\begin{displaymath}
 \left( \omega_1 \wedge \ldots \wedge \omega_k \right) (x_1, \ldots, x_k) = \textrm{det}\big(\omega_i(x_j) \big)
\end{displaymath}



\begin{displaymath}
 \Lambda^k(V^*) = \underbrace{V^\ast \wedge \ldots \wedge V^\ast}_{k times} = \{ \alpha : \underbrace{V \times \ldots \times V}_{k times} \rightarrow \KField \st \textrm{multilinear, alternating}\}
 \} \subset \TensorSpace^0_k(V) = \TensorSpace^k_0(V^\ast)
\end{displaymath}

\begin{displaymath}
 \Lambda(V^*) = \bigoplus_{k=0}^{n} \Lambda^k(V^\ast)
\end{displaymath}


$$-- Remark --$$
A completely similar construction could be achieved defining the symmetrizing product instead of the antisymmetrizing! (per ora non esplicito questa cosa.)

$\Lambda(V^*)\subset \TensorSpace(V) $ is a linear subspace but not a subalgebra. $(\Lambda(V^\ast), \wedge)$ constitutes an algebra called exterior or Grassman algebra.


\subsection{basis}
Fixed a basis $\{e_i\}$ in $V$ it's automatically obtained a basis on the zoo of tensor spaces

$e^\ast_i = e^i \st e^i(e_j) = \delta^i_j $ on the dual

$e_{i_1} \otimes \ldots \otimes e_{i_p} \otimes e^{j_1} \otimes \ldots \otimes e^{j_q}$ on tensors

$e_1 \wedge \ldots \wedge e_k $ in $\Lambda^k(V^*)$


$$--- oss ---$$
Ho qualche d extern in astratto?

	\bibliographystyle{plain}
	\bibliography{../DispensariumBibliography}


\end{document}
