\documentclass[a4paper,12pt]{scrartcl}    %attenzione! uso scratctl che permette sottotitolo


\usepackage{hyperref} %linktoUrl


\begin{document}

%  Titolo
	\title{MEMENTO:}
	\subtitle{Personale,.. sulla logica di base della meccanica statistica (ispirata dal corso di proteine)}
	\author{Tony}
	\date{\today}
\maketitle

\section{Considerazioni Generali}
\begin{itemize}

\item In meccanica statistica si introduce lo \emph{stato macroscopico} relativo ad un sistema fisico.
Matematicamente questo e' caratterizzato da un insieme $\Omega$ sullo spazio della fasi $T*M$ detto  \href{http://it.wikipedia.org/wiki/Insieme_statistico}{ensamble} e da una distribuzione di probabilita' definita su M a supporto definito almeno in $\Omega$.

Quindi un macrostato e' ( e' specificato da) la coppia $( \Omega , P )$.

\item fissare un macrostato di un sistema significa affermare che non si conosce lo stato esatto in cui si trova il sistema (in questo contesto si ribattezza lo stato \emph{microstato}) ma si sa che il sistema puo' essere in un determinato stato dell'insieme $\Omega$  con probabilita' P.



\item[+] Cavilli: tutto cio' richiede la struttura di spazio misurabile sulla spazio delle fasi ( varieta' orientabili sono misurabili? lo spazio contangente di una varieta' liscia e' orientabile? e' naturalmente individuata una particolare misura in spazi di questo tipo?).

Come si interpreta una misura sullo spazio delle fasi? (oppure che interpretazione ha il volume di una regione di spazio delle fasi?)







\item Per sistemi chiusi (=il numero di microsistemi elementari che li costituiscono non cambiano. nella maggiorparte di casi un microsistema e' un punto materiale. cio' in accordo all'ipotesi riduzionista della fisica che afferma che tutto e' costituito da atomi qui intesi come sistemi fisici punti materiali) si identificano due classi di ensamble, \emph{microcanonico} e \emph{canonico}.

\item Per sistemi Aperti si parla di ensable grancanonico. notare pero' che in tal caso cambia la underlying structure! (Il T*M dell'inizio va bene per particelle fissate! devo fissare una struttura algebrica apposta i cui elementi possano essere stati per sistemi a N particelle , N" particelle eccetera)

\begin{tabular}{|r l || c | c |}
\hline
Proprieta' termodinamiche del sistema& & scambia energia con l'ambiente ? & scambia materia con l'ambiente? \\
\hline
\hline
Isolato & & no & no \\
\hline
Non Isolato & chiuso & si & no \\
 & Aperto & si & no \\
 \hline
\end{tabular}

\item A seconda delle proprieta' termodinamiche del sistema si assume che gli stati macroscopici siano di un certo tipo quindi gli ensamble associati siano strettamente appartenenti ad una delle classi precedenti.
I macrostati possibili associati ad una fissata termodinamica del sistema possono essere anch'essi paramatrizzati (faccio un analogo macroscopico delle coordinate di configurazione) da quantita' dette variabili macroscopiche.





\end{itemize}




\section{Ensamble \href{http://it.wikipedia.org/wiki/Insieme_microcanonico}{microCanonico}}
\begin{itemize}
\item Viene associato al macrostato ad Energia fissata. equivale percio' ad affermare che il sistema si trova in una configurazione ad energia $E$ ma non si sa con esattezza quale delle molte possibili.
(dubbio. viene associato al macrostato " Energia = E" oppure al macrostato " Energia $=E+ \delta E$ "?

\item \href{http://it.wikipedia.org/wiki/Ipotesi_ergodica}{\textbf{Postulato Equiprobabilita' a priori}} = La distribuzione di probabilita' associato a questo tipo di ensamble e' costante su $\Omega(E)$ curva di livello (sottovarieta'di T*M) ad energia costante ( $ H(p) = E \: \forall p \in \Omega(E)$) .




\end{itemize}



...  Tutto Migrato su OneNote



\end{document}