\documentclass[a4paper,12pt]{scrartcl}    %attenzione! uso scratctl che permette sottotitolo

%Per le Figure
\usepackage[english]{babel}
\usepackage{graphicx}

%simboli matematici strani quali unione disgiunta
\usepackage{amssymb}

%Scrivere Sotto i simboli
\usepackage{amsmath}

%Diagrammi Commutativi
\usepackage{tikz}
\usetikzlibrary{matrix}

%Il simbolo di Identità
\usepackage{dsfont}

%Per riflettere i simboli...
\usepackage{graphicx}


%link iNTERNET
\usepackage{hyperref}

%Enumerate with letters
\usepackage{enumerate}

%Slash over letter
\usepackage{cancel}

%Usare bibiliografia bibtex
%\bibliographystyle{plain}

%Danger sign
\usepackage{fourier}

%:=
\usepackage{mathtools}



%Common symbols
%Common math symbols
	%Number fields
		\newcommand{\Real}{\mathbb{R}}
		\newcommand{\Natural}{\mathbb{N}}
		\newcommand{\Relative}{\mathbb{Z}}
		\newcommand{\Rational}{\mathbb{Q}}
		\newcommand{\Complex}{\mathbb{C}}
	
%equality lingo
	%must be equal
		\newcommand{\mbeq}{\overset{!}{=}} 

% function
	%Domain
		\newcommand{\dom}{\mathrm{dom}}
	%Range
		\newcommand{\ran}{\mathrm{ran}}
	

% Set Theory
	% Power set (insieme delle parti
		\newcommand{\PowerSet}{\mathcal{P}}

%Differential Geometry
	% Atlas
		\newcommand{\Atlas}{\mathcal{A}}
	%support
		\newcommand{\supp}{\textrm{supp}}

	
	
%Category Theory
	%Mor set http://ncatlab.org/nlab/show/morphism
%		\newcommand{\hom}{\textrm{hom}}

%Geometric Lagrangian Mechanics
	% Kinematic Configurations
		\newcommand{\Conf}{\mathtt{C}}
	%Solutions Space
		\newcommand{\Sol}{\mathtt{Sol}}
	%Lagrangian class
		\newcommand{\Lag}{\mathsf{Lag}}
	%Lagrangiana
		\newcommand{\Lagrangian}{\mathcal{L}}
	%Data
		\newcommand{\Data}{\mathsf{Data}}
	%unique solution map
		\newcommand{\SolMap}{\mathbf{s}}
		
		
		
%Peierls (per non sbagliare più)
		\newcommand{\Pei}{Peierls}


%Temporaneo, Aggiunta della mia classe teorem... Deve diventare un pacchetto!
\input{../Latex-Theorem/TheoremTemplateToninus.tex}

\newcommand{\Op1}{\spadesuit}
\newcommand{\Op2}{\clubsuit}


\begin{document}

%  Titolo
	\title{Prolegomena on (some) Abstact Algebras}
	\subtitle{}
	\author{Tony}
	\date{\today}
\maketitle

\begin{abstract}
	(molto basato su wiki)

\end{abstract}

%  Indice
\tableofcontents

%-_-_-_-_-_-_-_-_-_-_-_-_-_-_-_-_-_-_-_-_-_-_-_-_-_-_-_-_-_-_-_-_-_-_-_-_-_-_-_-_-_-_-_-_-_-_-_-_-_-_-_-_-_-_-_
\newpage
\section{Zoo of Algebraic Structure}

\section{Single Operation structures (Magmas)}
Consider two set $M, A$ and a binary operation, i.e. a function:
\begin{displaymath}
	\Op1 : M \times M \rightarrow A
\end{displaymath}

Operation $\Op1$ is said:
\begin{itemize}
\item \emph{Closed} if $\Ran(\Op1)\subset M$
	\begin{equation}
		( x \Op1 y ) \in M \qquad \forall x,y\in M
	\end{equation}
\item \emph{Associative}
\item \emph{Unital}
	\begin{equation}
	
	\end{equation}
\item \emph{Exists Inverse:}
	\begin{equation}
	
	\end{equation}
\item \emph{Commutative}
	\begin{equation}
	
	\end{equation}
\end{itemize}


\end{document}
