\documentclass[a4paper,12pt]{scrartcl}    %attenzione! uso scratctl che permette sottotitolo

%Per le Figure
\usepackage[english]{babel}
\usepackage{graphicx}

%simboli matematici strani quali unione disgiunta
\usepackage{amssymb}

%Scrivere Sotto i simboli
\usepackage{amsmath}

%Diagrammi Commutativi
\usepackage{tikz}
\usetikzlibrary{matrix}

%Il simbolo di Identità
\usepackage{dsfont}

%Per riflettere i simboli...
\usepackage{graphicx}


%link iNTERNET
\usepackage{hyperref}

%Enumerate with letters
\usepackage{enumerate}

%Slash over letter
\usepackage{cancel}

%Usare bibiliografia bibtex
%\bibliographystyle{plain}

%Danger sign
\usepackage{fourier}

%:=
\usepackage{mathtools}



%Common symbols
%Common math symbols
	%Number fields
		\newcommand{\Real}{\mathbb{R}}
		\newcommand{\Natural}{\mathbb{N}}
		\newcommand{\Relative}{\mathbb{Z}}
		\newcommand{\Rational}{\mathbb{Q}}
		\newcommand{\Complex}{\mathbb{C}}
	
%equality lingo
	%must be equal
		\newcommand{\mbeq}{\overset{!}{=}} 

% function
	%Domain
		\newcommand{\dom}{\mathrm{dom}}
	%Range
		\newcommand{\ran}{\mathrm{ran}}
	

% Set Theory
	% Power set (insieme delle parti
		\newcommand{\PowerSet}{\mathcal{P}}

%Differential Geometry
	% Atlas
		\newcommand{\Atlas}{\mathcal{A}}
	%support
		\newcommand{\supp}{\textrm{supp}}

	
	
%Category Theory
	%Mor set http://ncatlab.org/nlab/show/morphism
%		\newcommand{\hom}{\textrm{hom}}

%Geometric Lagrangian Mechanics
	% Kinematic Configurations
		\newcommand{\Conf}{\mathtt{C}}
	%Solutions Space
		\newcommand{\Sol}{\mathtt{Sol}}
	%Lagrangian class
		\newcommand{\Lag}{\mathsf{Lag}}
	%Lagrangiana
		\newcommand{\Lagrangian}{\mathcal{L}}
	%Data
		\newcommand{\Data}{\mathsf{Data}}
	%unique solution map
		\newcommand{\SolMap}{\mathbf{s}}
		
		
		
%Peierls (per non sbagliare più)
		\newcommand{\Pei}{Peierls}


%Temporaneo, Aggiunta della mia classe teorem... Deve diventare un pacchetto!
\input{../Latex-Theorem/TheoremTemplateToninus.tex}




\begin{document}

%  Titolo
	\title{Prolegomena on (some) Riemmanian Geometry}
	\subtitle{Rieammanian manifolds, geodesic,Jacobi fields...}
	\author{Tony}
	\date{\today}
\maketitle

\begin{abstract}
	Cerco di scrivere la teoria di geometria differenziale minimale necessaria per la tesi.
	Al momento non c'è un ordine logico per stabilito (parlo di geodetiche in senso variazionale prima di dire cosa sono le variazioni.... )
	
	Fonti Principali di ispirazione sono :
	\begin{itemize}
		\item Jost, J., 2005. Riemannian Geometry and Geometric Analysis, Berlin/Heidelberg: Springer-Verlag. Available at: http://link.springer.com/10.1007/3-540-28891-0 [Accessed January 10, 2015].
		\item Abate, M., Tovena, F., 2011. Geometria Differenziale, Milano: Springer Milan. Available at: http://link.springer.com/10.1007/978-88-470-1920-1 [Accessed January 10, 2015].
		\item Abraham, R. et al., 1978. Foundations of mechanics II., Available at: http://www.ams.org/publications/authors/books/postpub/author-pages [Accessed January 10, 2015].
	\end{itemize}
\end{abstract}

%  Indice
\tableofcontents


\newpage
% Cio' Che Segue Andrà a finire in un  chapter

%-_-_-_-_-_-_-_-_-_-_-_-_-_-_-_-_-_-_-_-_-_-_-_-_-_-_-_-_-_-_-_-_-_-_-_-_-_-_-_-_-_-_-_-_-_-_-_-_-_-_-_-_-_-_-_
\newpage
\section{Reprise in Riemannian Geometry}
In what follows we present a brief review of the most important result in Riemannian geometry necessary for a better understanding of the geodesic problem.

\subsection{Definition of (pseudo)-Riemannian manifold}
\begin{definition}[(Pseudo-)Riemannian manifold]
\end{definition}

\begin{notationfix}
 \emph{Metric signature} \emph{Lorenz manifold}.
\end{notationfix}

\begin{theorem}
$\forall M$ is Riemannianizable.
\end{theorem}

\subsection{Riemannian manifolds as a category.}
\begin{definition}[Local isometry]
\end{definition}

\begin{definition}[Isometry]
\end{definition}

\begin{definition}[Killing fields]
\end{definition}

\subsection{Riemannian as a measure space.}
\begin{definition}[Riemannian volume form]
\end{definition}

\begin{theorem}
$\forall M$ orientable $\exists1!$ Riemannian volume form.
\end{theorem}

\begin{observation}
For an insight on the connection between volume form and measure theory see for example \cite{Abraham1978}.
\end{observation}

\subsection{Tangent bundle of a Riemannian manifold.}
\begin{observation}
$g$ could be seen as a 2-forms (section $\in \Gamma(T^2_0(M))$
\end{observation}

\begin{definition}[$\flat$ $\sharp$ operator]
	(sarebbero abbassamento e innalzamento)
\end{definition}

\begin{theorem}
 On Riemannian manifold $M$ $TM$ is a structure manifold of structure group $G= O(d)$.
 \\
 If $M$ is also orientable $G= SO(d)$.
\end{theorem}
\begin{proof}
 See \cite{Jost2005} Lemma $1.5.2$ and $1.5.3$.
\end{proof}

\subsection{Riemannian as a metric space.}
 See \cite{Abate2011} pag $383-385$  and \cite{Jost2005} pag $15-17$.
 
 Consider the space $C^\infty([a,b],M)$ of the smooth parametrized curves from a closed interval of the real line.
 We define the following functional:
 \begin{definition}[Length functional]
 	\begin{displaymath}
 		L(\gamma) \coloneqq \int_a^b \left\Vert \frac{d \gamma}{dt} (t)\right\Vert dt
 	\end{displaymath}
\end{definition} 
 
\begin{definition}[Energy functional]
  	\begin{displaymath}
 		E(\gamma) \coloneqq \int_a^b \left\Vert \frac{d \gamma}{dt} (t)\right\Vert^2 dt
 	\end{displaymath}
\end{definition} 
Are define them together because are linked trough an inequality:
\begin{proposition}
	$\forall \gamma : [a,b] \rightarrow \Real$ smooth parametrized curve:
	\begin{equation}
		L(\gamma)^2 \leq 2(b-a)E(\gamma)
	\end{equation}
	equality holds iff $\left\Vert \frac{d \gamma}{dt} (t)\right\Vert \equiv const $
\end{proposition}	
\begin{proof}
	 See \cite[Lemma $1.4.2$ ]{Jost2005}.
\end{proof}
	
\begin{observation}
	Former concepts are extended slavishly to every piecewise smooth curve.
\end{observation}

\begin{definition}[Distance between two points]
	Function $d:M\times M \rightarrow \Real$
	\begin{displaymath}
		d(p,q)\coloneqq \inf\big\{L(\gamma) \big\vert \; \gamma:[a,b]\rightarrow \Real \; \textrm{\small piecewise smooth},\, \gamma(a)=p,\, \gamma(b)=q  \big\}
	\end{displaymath}
\end{definition}
\begin{observation}
	Distance is well defined for all pair of point Iff the manifold is connected. 
\end{observation}

\begin{proposition}
The distance function satisfies the axioms of metric:
\begin{itemize}
	\item  \emph{non-negative:}\quad$d(p,q)\geq 0 \; \forall p,q\in M \qquad d(p,q)>0 \; \forall p\neq q$
	\item \emph{simmetric:}\quad$d(p,q) = d(q,p)$ 
	\item  \emph{triangle inequality:}\quad$d(p,q) \leq d(p,r) + d(r,q) \; \forall p,q,r \in M$
\end{itemize}
\end{proposition}
\begin{proof}
	 See \cite[Lemma $1.4.1$ ]{Jost2005}.
\end{proof}

\begin{corollary}
	 The "balls" topology of $M$ induced by the distance function $d$ coincides with the original manifold topology of $M$.
\end{corollary}
\begin{proof}
		 See \cite[corollary $1.4.2$ ]{Jost2005}.
\end{proof}

\subsection{Connection structure on a Riemannian manifold.}
 Connection is a rather general concept definable on any smooth bundle. \footnote{In this abstract context connection takes the name of \emph{Erhesmann's connection}.}
 
 On vector bundle we can identify a special kind of connection structure compatible with the vector space structure.\footnote{which takes its name from \emph{Koszul} for distinguish it from the above.}
 There are several equivalent presentation of this concept, each of them stress the importance of one of the many devices carried by this superstructure, for example:
 \begin{itemize}
 	\item Derivative of section.
 	\item Parallelism and parallel transportation.
 	\item Specification of an unique horizontal lift among all.
 \end{itemize}
 
 Regarding the Riemannian manifolds we're not interested in connections on general vector bundle but instead to those on the tangent bundle, called \emph{Linear Connection}.
 There's an infinity of such connection but on (pseudo-)Riemannian manifold it's possible to find a natural prescription that allows us to identify only one among these, called \emph{Levi-Civita Connection}.

Consider $(M,g)$ pseudo-Riemannian manifold.

\begin{definition}[Linear Connection]
 Map $\nabla : \Gamma^\infty(\tau_M) \times \Gamma^\infty(\tau_M) \rightarrow \Gamma^\infty(\tau_M)$,
 we write $(X,Y)\mapsto \nabla_X Y \qquad \forall X,Y \in \Gamma^\infty(\tau_M)$.
 Such that:
 \begin{enumerate}[(a)]
  \item $\nabla_X Y $ is $C^\infty(M)-$linear in $X$ variable.
   \begin{displaymath}
   \nabla_{f X_1 + g X_2}Y = f\nabla_{X_1} Y + g \nabla_{X_2} Y \qquad \forall f,g \in C^\infty(M)
   \end{displaymath}
 
  \item
  
  \item
 
 \end{enumerate}


\end{definition}

\subsection{Temp: Memento covariant Derivative}
\url{http://www.physicspages.com/2014/01/02/covariant-derivative-of-the-metric-tensor/}
$\nabla\mu g_{\alpha \beta} = 0$

\subsection{Curvature on Riemannian manifold.}
%-_-_-_-_-_-_-_-_-_-_-_-_-_-_-_-_-_-_-_-_-_-_-_-_-_-_-_-_-_-_-_-_-_-_-_-_-_-_-_-_-_-_-_-_-_-_-_-_-_-_-_-_-_-_-_
\newpage
\section{Geodesic}
\subsection{Common approach to the Geodesic}
	On a manifold endowed with a affine connection a geodesic is defined as a curve "everywhere parallel to itself" providing a generalization of \emph{straight line}.
	\begin{definition}[Geodesic]
		A curve \danger\footnote{Devo dire smooth o piecewise? }
		$\gamma:[a,b]\rightarrow M$ such that:
		\begin{equation}
			\nabla_{\dot{\gamma}}\dot{\gamma} =0
		\end{equation}
		where $\dot{\gamma}^\mu \coloneqq \frac{d \gamma^\mu}{d t}$ is the tangent vector to the curve.
	\end{definition}
	\begin{notationfix}
		In local chart the previous equation assume the popular expression:
		\begin{equation}\label{GeodesicEquation}
			\ddot{\gamma}^i + \Gamma^i_{\, j k} \dot{\gamma}^j \dot{\gamma}^k = 0
		\end{equation}
		Where $ \Gamma^i_{\, j k}$ is the coordinate representation of the Christoffel symbols of the connection.
	\end{notationfix}
	\hspace{5mm}


%-_-_-_-_-_-_-_-_-_-_-_-_-_-_-_-_-_-_-_-_-_-_-_-_-_-_-_-_-_-_-_-_-_-_-_-_-_-_-_-_-_-_-_-_-_-_-_-_-_-_-_-_-_-_-_
\newpage
\section{Review of physics application of geodesic problem.}
Essentially \cite{Abraham1978}.
\vspace{6mm}
A lot of mechanics systems can be regard as geodesic problem.

\subsection{Preliminary remarks: Geometrical encoding of classical mechanics.}
    sistemi hamiltoniani

    sistemi lagrangiani
    
\subsection{Particle on Riemannian manifold under a position dependant potential.}	
	fomm pag 226-228 + 231-233       teo 3.71
 
    
    
\subsection{Relativistic particle.}
	\footnote{For an extension of this process to costrained, dissipitative or ergodic systems see fom cap 3.7}


%-_-_-_-_-_-_-_-_-_-_-_-_-_-_-_-_-_-_-_-_-_-_-_-_-_-_-_-_-_-_-_-_-_-_-_-_-_-_-_-_-_-_-_-_-_-_-_-_-_-_-_-_-_-_-_
\newpage
\section{Jacobi Fields}

\subsection{Preliminary remarks: Variation of curve.}
Let $\sigma:[a,b]\rightarrow M$ a piecewise regular curve on smooth manifold $M$.

 \begin{definition}[Variation of Curve]
  Variation of curve $\sigma$ is a continuous application $\Sigma: (-\varepsilon, \varepsilon) \times [a,b] \rightarrow M$ such that
  \begin{itemize}
   \item $\sigma_s = \Sigma(s, \cdot)$ is a piecewise regular curve $\forall s \in  (-\varepsilon, \varepsilon)$.
   \item $\sigma_0 = \sigma$.
   \item $\exists$ a partion $ a= t_0 < t_1 < \ldots < t_k=b$ of $[a,b]$ such that 
   $$ \Sigma \big \vert_{(-\varepsilon, \varepsilon) \times [t_{j-1},t_{j}]}\; \in \; \mathcal{C}^\infty(\mathbb{R}^2; M)$$.
  \end{itemize}
 \end{definition}
 
 \begin{notationfix}
  Regarding one entry as a variable and the other as a parameter we can see that $\Sigma$ determine two family of curves:
  \begin{itemize}
   \item $\sigma_s(\cdot)= \Sigma(s, \cdot)$ is a family of piecewise regular curves called \emph{principal curves}.
   \item $\sigma^t(\cdot)= \Sigma(\cdot , t)$ is a family of regular curves called \emph{transverse curves}.
  \end{itemize}
  Curves in a family have a common parametrization.
 \end{notationfix}
 
 \begin{notationfix}
  A variation is called \emph{proper} if the endpoints stay fixed, i.e.
  \begin{displaymath}
   \sigma_s(a)= \sigma(a) \; \wedge \; \sigma_s(b) = \sigma(b) \qquad \forall s\in (-\varepsilon, \varepsilon)
  \end{displaymath}
 \end{notationfix}

Fields over a variation $\Sigma$ of a curve $\sigma$ are defined as follows:

 \begin{definition}[Vector field along a variation]
  Is a collection $X= \{X_j \}$ of smooth applications $X_j:(-\varepsilon, \varepsilon) \times [t_{j-1},t_{j}]\rightarrow TM$ \footnote{Associate to a subdivision of $ a= t_0 < t_1 < \ldots < t_k=b$ of $[a,b]$. } such that:
  \begin{displaymath}
   X_j (s,t) \in T_{\Sigma(s,t)}M \qquad \forall (s,t) \in (-\varepsilon, \varepsilon) \times [t_{j-1},t_{j}]
   \quad \forall j= 1, \ldots, k
  \end{displaymath}
 \end{definition}
 

Principal and  transverse curves define two special Vector fields along the variation:
\begin{definition}[Tangent fields of the variation]\label{Def: Variation Field}
 \begin{displaymath}
  S(s,t) = (\sigma^t)'(s) = \textrm{d}\Sigma_{(s,t)} \big( \frac{\partial}{\partial s} \big) = 
  \frac{\partial \Sigma}{\partial s} (s,t)
 \end{displaymath}
 for all $(s,t) \in (-\varepsilon, \varepsilon) \times [a,b] $.
 \begin{displaymath}
    T(s,t) = (\sigma_s)'(t) = \textrm{d}\Sigma_{(s,t)} \big( \frac{\partial}{\partial t} \big) = 
  \frac{\partial \Sigma}{\partial t} (s,t)
 \end{displaymath}
 for all $(s,t) \in (-\varepsilon, \varepsilon) \times [t_{j-1},t_{j}] $ and $j=1, \ldots, k-1$ 
 where we have choose  a subdivision $ a= t_0 < t_1 < \ldots < t_k=b$ associated to $\Sigma$.
\end{definition}
 
\begin{notationfix}
 $V= S(0,\cdot) \in \mathfrak{X}(\sigma)$ takes the special name of \emph{variation field of $\Sigma$}.
\end{notationfix}

There's an importation relation between continuous field on a curve and variation:
\begin{proposition}
 For all continuous field $V$ along a piecewise regular curve $\sigma$ can be found a variation 
 $\Sigma$ with variation field $V$.
 \footnote{Vice versa follows from the continuity of the variation field.}
\end{proposition}
\begin{proof}
 See \cite{Abate2011} Lemma $7.2.12$ .
\end{proof}

\vspace{8mm}
Let now $M$ be a d-dimensional Riemannian manifold with Levi-Civita connection $\nabla$.
The tangent fields of a variation are strictly connected to the curvature of M.
We need a lemma:

\begin{lemma}
 For all rectangle $(-\varepsilon,\varepsilon) \times [t_{j-1}, t_j] \in \mathbb{R}^2$ 
 on which $\Sigma$ is $\mathcal{C}^{\infty}$ we have:
 \begin{displaymath}
   D_S T = D_T S
 \end{displaymath}
 where $D_S$ is the covariant derivative along the transverse curves and $D_T$ over the principal curves.
\end{lemma}
\begin{proof}
 See \cite{Abate2011} Lemma $7.2.13$ .
\end{proof}
The crucial result is what follows:
\begin{proposition}
 For all vector field $V$ along a variation $\Sigma$ we have:
 \begin{displaymath}
   D_S D_T V - D_T D_S V = R( S, T) V
 \end{displaymath}
 for all rectangle $(-\varepsilon,\varepsilon) \times [t_{j-1}, t_j] \in \mathbb{R}^2$ on which $\Sigma$ is $\mathcal{C}^{\infty}$ .
 \footnote{$ R(S, T)$ is the curvature endomorphism evaluated on the tangent vector fields on the variation.}
\end{proposition}
\begin{proof}
 See \cite{Abate2011} Lemma $8.2.3$ .
\end{proof}

 (References: \cite{Abate2011} page 386-387 + 420-421 ; \cite{Jost2005} page 171)


\subsection{Formal Definition}
The concept of \emph{Jacobi Field} is closely related to variations of geodesic curves.
In fact it describes the difference between the geodesic and an "infinitesimally close" geodesic. In other words, the Jacobi fields along a geodesic form the tangent space to the geodesic in the space of all geodesics. 
\\
Let $\gamma:[a,b]\rightarrow M$ be a geodesic of the Riemannian manifold $M$. We can consider a special class of variations:

\begin{definition}[Geodesic variation]
Is a smooth variation $\Sigma:(-\varepsilon, \varepsilon) \times [a,b] \rightarrow M$ such that all the principal curves $\gamma_s (\cdot) = \Sigma(s,\cdot)$ are also geodesic.\footnote{In other words $\Sigma$ determines a smoothly variable family of geodesic.}
\end{definition}

\begin{proposition}
 Fixing two tangent vector over a point $p= \gamma(a)$ on the geodesic $\gamma$ univocally determines 
 a geodesic variation of $\gamma$.
\end{proposition}
\begin{proof}
 See \cite{Abate2011} Lemma $8.2.5$  or \cite{Jost2005} Lemma $4.2.3$.
\end{proof}

\begin{definition}[Jacobi Fields]
 Is a field $J \in \mathfrak{X}(\gamma)$ over a geodesic $\gamma$ such that:
 \\
 $\exists \Sigma$ geodesic variation such that $J=V$ represent its variation field \footnote{As defined under (def \ref{Def: Variation Field}).}.
\end{definition}
The following proposition determines an equivalent (analytical) definition of Jacobi field:

\begin{proposition}
$$ $$
 $J \in \mathfrak{X}(\dot{\gamma})$ is a jacobi field iff:
 \begin{displaymath}
   \nabla_{\frac{\textrm{d}}{\textrm{d}t}} \nabla_{\frac{\textrm{d}}{\textrm{d}t}} J  + R(X,\dot{\gamma}) \gamma' = 0
 \end{displaymath}
\end{proposition}

\begin{notationfix}
The vector space of all Jacobi fields on the geodesic $\gamma$ is denoted $\mathcal{J}(\gamma)$.
\end{notationfix}
\begin{notationfix}
 $J \in J(\gamma)$ is called \emph{proper} if $J_0(t)\perp \dot{(\gamma)}(t)$.
\\
 $\mathcal{J}(\gamma)$ indicates the vector space of all proper Jacobi fields.
\end{notationfix}

\begin{proposition}
Every killing field $X$ on $M$ is a Jacobi Field along any geodesic in $M$.
\end{proposition}
\begin{proof}
 See \cite{Jost2005} Corollary $4.2.1$.
\end{proof}

%-_-_-_-_-_-_-_-_-_-_-_-_-_-_-_-_-_-_-_-_-_-_-_-_-_-_-_-_-_-_-_-_-_-_-_-_-_-_-_-_-_-_-_-_-_-_-_-_-_-_-_-_-_-_-_

%Temporaneo BiblioGrafia
\bibliography{NonSync-Tesi-Bibiliografia}


%-_-_-_-_-_-_-_-_-_-_-_-_-_-_-_-_-_-_-_-_-_-_-_-_-_-_-_-_-_-_-_-_-_-_-_-_-_-_-_-_-_-_-_-_-_-_-_-_-_-_-_-_-_-_-_
\newpage
\section{Closing Thoughts}

\subsection{Eliminata}
\begin{itemize}

\item non messa la definizione dei campi continui e l'osservazione che S  è sempre continuo mentre $T$ può non esserlo (\cite{Abate2011} pag 420).

\item sono stato ambiguo quando parlo di campi lungo la curva.. sulla continuità o meno (vedere abate pag 387)

\item non mi è ancora chiaro l'utilità dei jacobi fields... Vediamo le possibilità:
	\begin{itemize}
	\item Dice Abate a pag. $411$ i Jacobi sono lo strumento principale per stabilire una relazione fra curvatura e topologia.
	\item Dice Jost a pag. $183$ che le Jacobi equation sono una linearizzazione dell'equazione delle geodetiche.
	\item Jost a pag $183-186$ esplora il legame tra $J$  e le mappe esponenziali.
	\item Jost nel capitolo $4.3$ e  Abate a pag $424 + 433-435$ parlano del legame con i punti coniugati e morse theory.
	\end{itemize}

\item Discorso della index form come azione le cui equazioni eulero lagrange determinano l'equazione geodetica. (fonte Jost pag $177-179$).

\item Discorso Decomposizione dei Jacobi field in campi orizzonatali e verticali (fonte Jost pag $180-181$, \url{http://en.wikipedia.org/wiki/Jacobi_field}.


\end{itemize}



%-_-_-_-_-_-_-_-_-_-_-_-_-_-_-_-_-_-_-_-_-_-_-_-_-_-_-_-_-_-_-_-_-_-_-_-_-_-_-_-_-_-_-_-_-_-_-_-_-_-_-_-_-_-_-_
%Temporaneo BiblioGrafia







\end{document}
