%	Appunto di carralleta sui concetti di derivazione in geometria differenziale!
%
%


\documentclass[a4paper,12pt]{scrartcl}    %attenzione! uso scratctl che permette sottotitolo

% PREAMBLE %%%%%%%%%%%%%%%%%%%
\usepackage{hyperref}
\usepackage[subpreambles=true]{standalone}
 
\usepackage{amssymb}
\usepackage{amsmath}
\usepackage{amsfonts}

\renewcommand*{\thefootnote}{\alph{footnote}}


\usepackage[tensors,diffgeo]{../Math-Symbols-List/toninus-math-symbols}





%%%%%%%%%%%%%%%%%%%


\begin{document}

%  Titolo
	\title{MEMENTO:}
	\subtitle{Derivazione in Varieta' differenziabili}
	\author{Tony}
	\date{\today}
\maketitle

%introduzione
bla bla

%  Indice
\tableofcontents

\section{Work in Progress!}
(substitute $\partial_i f $ with $\dfrac{\partial f}{\partial x^i}$ in order to avoid confusion between natural basis vector and usual partial derivative operator acting on function$f$)

		\fbox{
		  \parbox{\textwidth}{
				A \emph{graded derivation} of $\Omega(M)$ is a degree $k$ linear operator $A$ on $\Omega(M)$ such that
				\begin{equation}
					A (\omega \wedge \eta) = A(\omega) \wedge \eta + (-1)^{kp} \omega \wedge A(\eta) \qquad \forall \omega \in \Omega^k(M) , \; \eta \in \Omega^\cdot(M)
				\end{equation}
		  }
		}

		\fbox{
		  \parbox{\textwidth}{
			An \emph{($I$-)graded vector space} $V$ is a vector space that can be written as a direct sum of subspaces indexed by elements $i$ of set $I$:
			\begin{displaymath}
				V = \bigoplus_{i \in I} V_i
			\end{displaymath}
		  }
		}

		\fbox{
		  \parbox{\textwidth}{
			An \emph{(p-)graded linear operator} $A$ is a linear operator on $V$ such that: 
			\begin{displaymath}
				V = \bigoplus_{i \in I} V_i
			\end{displaymath}
		  }
		}

		\fbox{
		  \parbox{\textwidth}{
			An \emph{(graded-)commutatative ($I$-)graded algebra} $(V, \wedge)$ is an algebra over $\Real$\footnote{Could be any ring in principle} such that :
			\begin{displaymath}
				\mapsto v^{(i)} \wedge w = w \wedge v^{i} \qquad \forall v^{(i)} \in V^{(i)}, w \in V
			\end{displaymath}
		  }
		}		

\section{Derivata in Astratto}
\url{http://en.wikipedia.org/wiki/Derivation_%28differential_algebra%29]}



\section{Derivata di Lie}
\href{run:d:~/Downloads/FoM2.pdf}{link}
Jurgen Jost, Riemannian Geometry and Geometric Analysis, (2002) Springer-Verlag, Berlin ISBN 3-540-42627-2 See section 1.6.



\section{Derivata esterna}
Capitolo 2.4 di Abraham Marsden

\section{Derivata Covariante}
Capitolo 2.7 di Abraham Marsden
		\\
 		ChecK Covariant derivative on forms:
		\begin{displaymath}
			\nabla_X \ExtD x^\mu = \left[ \nabla_X \ExtD x^\mu \right]_\alpha \ExtD x^\alpha
		\end{displaymath}
		\begin{displaymath}
			\left[ \nabla_X \ExtD x^\mu \right]_\alpha =
			\left\langle \nabla_X \ExtD x^\mu  \right\vert \left. \partial_\alpha \right\rangle =
			X^\nu \left\langle \nabla_\nu \ExtD x^\mu  \right\vert \left. \partial_\alpha \right\rangle =			
			X^\nu \left(\nabla_\nu\left \langle  \ExtD x^\mu  \right\vert \left. \partial_\alpha \right\rangle - \left\langle \ExtD x^\mu  \right\vert \left. \nabla_\nu \partial_\alpha \right\rangle \right) =	
			X^\nu \left( - \Gamma^\mu_{\, \nu \, \alpha} \right)
		\end{displaymath}   

\section{Vettori tangenti come derivazione}
Capitolo 2.2 di Abraham Marsden



\end{document}