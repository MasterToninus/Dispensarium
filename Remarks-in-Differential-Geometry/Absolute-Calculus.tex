\documentclass[a4paper,12pt]{scrartcl}
\usepackage{pdflscape}
\usepackage{tabularx}
\usepackage{array}
\setlength{\extrarowheight}{6mm}
\usepackage{geometry}
 \geometry{
 a4paper,
 total={185mm,277mm},
 left=5mm,
 right=5mm,
 top=5mm,
 bottom=5mm,
 }
 
\usepackage[subpreambles=true]{standalone}
 
\usepackage{amssymb}
\usepackage[leqno]{amsmath}
\usepackage{amsfonts}

\usepackage[symbol]{footmisc}
\renewcommand*{\thefootnote}{\fnsymbol{footnote}}



\usepackage{arydshln}

\usepackage[tensors,diffgeo]{../Math-Symbols-List/toninus-math-symbols}
%https://tex.stackexchange.com/questions/48980/whole-page-table-with-tabularx

\begin{document}
  \begin{landscape}
    \thispagestyle{empty}
    \noindent
    \paragraph{ABSOLUTE DIFFERENTIAL CALCULUS}
    	\mbox{}\\
        $\quad$Suppose $M$ is a smooth manifold, $x^\mu$ a coordinate chart. Denote by $\Omega(M)$ the algebra of differential forms on $M$ and by $\mathfrak{X}(M)$ the $C^\infty(M)$-module of vector fields.  \\
    \vspace{5mm}
    \begin{tabularx}{\linewidth}{|c|X|c|c|c|c|}
      \hline
     	  & $f \in C^\infty(M)$ & $\ExtD x^\mu \in \Omega^1(M)$ & $\partial_\mu \in \mathfrak{X}(M)$ & $T_1 \otimes T_2$ & $\omega^{(k)} \wedge \beta$ \\
      \hline
      	$\ExtD$ & $\left(\dfrac{\partial f }{\partial x^\nu} \right) \: \ExtD x^\nu$ & 0 & - & -  & $\left( \ExtD \omega \right) \wedge \beta + (-)^k \omega \wedge \left( \ExtD\beta \right) $\\
%
      	$\Lie_X$ & $X(f) = X^\nu \left(\dfrac{\partial f }{\partial x^\nu} \right)$ & $\Lie_X \ExtD x^\mu = \ExtD\left(X^\mu\right) =\left(\dfrac{\partial X^\mu }{\partial x^\nu} \right)  \ExtD x^\nu$ & $\Lie_X \partial_\mu = [X, \partial_\mu]$ & $\left(\Lie_X T_1\right) \otimes T_2 + T_1 \otimes \left(\Lie_X T_2 \right)$ & $\left( \Lie_X \omega \right) \wedge \beta + \omega \wedge \left(\Lie_X\beta \right)$ \\
%
      	$\iota_X$  & $0$ & $\iota_X \ExtD x^\mu = \ExtD x^\mu (X) = X^\mu$  & $0$  & $\left(\iota_X T_1\right) \otimes T_2 + T_1 \otimes \left(\iota_X T_2 \right)$ & $\left( \iota_X \omega \right) \wedge \beta + (-)^k \omega \wedge \left( \iota_X\beta \right) $ \\
%
      	\cdashline{4-5}
      	$g^\ast$  & $g^\ast \left(f\right) = g \circ f $ & $ g^\ast \left(\ExtD x^\mu \right) = \ExtD\left(x^\mu \circ f \right)$ &$(g^{-1})_\ast \partial_\mu$ \quad  \footnotemark[3]  & $g^\ast \left( T_1\right) \otimes g^\ast \left( T_2\right)$ \quad  \footnotemark[3]  & $g^\ast\left(\omega\right) \wedge g^\ast \left( \beta \right)$\\ 
      	\hdashline
      	%$\nabla_X$ & $\nabla_X f = X(f) $ & & & & \\
      	$\nabla_X$ & $\nabla_X f = X(f) = X^\nu \left(\dfrac{\partial f }{\partial x^\nu} \right)$ & $\nabla_X \ExtD x^\mu = X^\nu \left( - \Gamma^\mu_{\, \nu \, \alpha} \right) \ExtD x^\alpha$ & $\nabla_X \partial_\mu =X^\nu \Gamma^\alpha_{\, \nu \, \mu} \partial_\alpha$ & $\left(\nabla_X T_1\right) \otimes T_2 + T_1 \otimes \left(\nabla_X T_2 \right)$  & $\left( \nabla_X \omega \right) \wedge \beta + \omega \wedge \left(\nabla_X\beta \right)$\\%& & & & & \\
            	\hdashline
    \end{tabularx}


	\begin{minipage}[c][.45\textheight]{0.46 \linewidth}
		The first four operation are naturally defined on every smooth manifold.
		\fbox{
		  \parbox{\textwidth}{
				\begin{displaymath}
			    	\mathcal{T}(M) = \left(\bigoplus_{l,k=0}^\infty
			    	 T^k_l(M), 
			    	 \otimes \right) 
			    	 \qquad \textrm{\emph{Tensor Algebra} on M}
			    \end{displaymath}
			    is a bi-graded algebra over the ring $C^\infty(M)$.
		  }
		}
		\vfill
		\fbox{
		  \parbox{\textwidth}{
		  	$(\mathfrak{X}(M) , [-,-])$ form a Lie algebra over $\mathbb{R}$:
		  	\begin{align}
		  		[X,Y] &= -[Y,X] \\
		  		[aX + bY, Z] &= a[X,Z] + b[Y,Z] \\
		  		[[X,Y],Z] + [[Y,Z],X] + [[Z,X],Y] &= 0 
		  	\end{align}
		  }
		}
		\vfill
		\fbox{
		  \parbox{\textwidth}{	
		  	Lie bracket is defined by the following (redundant) equations:
		  	\begin{align*}
		  		\left[\partial_i, \partial_j\right] &= 0 \\
		  		\left[f \partial_i , g \partial_j\right] &= f \cdot (\partial_i g)\cdot\partial_j - g\cdot(\partial_j f )\cdot\partial_i \\
		  		\left[f X , g Y \right] &= f\cdot g\cdot[X,Y] + f\cdot X(g)\cdot Y - g\cdot Y(f)\cdot X  \\
		  		\left[X ,Y\right] &= \left( X^i \partial_i Y^j - Y^i\partial_i X^j \right) \partial_j		  				
		  	\end{align*}		
		  }
		}	
		
		
		
	\end{minipage}
	\hspace{1cm}
	\setcounter{equation}{0}
	\begin{minipage}[c][.5\textheight]{0.46 \linewidth}
		The last one is not natural, it is an additional structure given to $M$:\\
		\fbox{
		  \parbox{\textwidth}{
		  	\emph{Affine connection} 
		  	\begin{align*}
		  		\nabla: \mathfrak{X}(M) \times \mathfrak{X}(M) &\longrightarrow \mathfrak{X}(M) \\
		  				  												(X,Y) &\longmapsto \nabla_X Y		  	
		  	\end{align*}
			such that: % $\forall X,Y,Z \in \mathfrak{X}(M)$, $\forall a \in \Real$, $\forall f \in C^\infty(M)$:
		  	\begin{align}
		  		\nabla_{(X+ a Z)} Y &= \nabla_X Y + a \nabla_Z Y \\
		  		\nabla_{X} \left( Y+ a Z \right) &= \nabla_X Y + a \nabla_Y Z \\
		  		\nabla_{f\, X} Y &= f \, \nabla_X Y \\
		  		\nabla_{X}\left( f \, Y\right) &= \left(\nabla_X f \right)\, Y + f \, \left(\nabla_X Y \right)		
		  	\end{align}
		  }
		}
		\vfill
		\fbox{
		  \parbox{\textwidth}{
		  	\begin{displaymath}
		  		\Gamma^\alpha_{\, \mu \, \nu} = \ExtD x^\alpha \left( \nabla_{\mu} \partial_\nu \right)
		  					    	 \qquad \textrm{\emph{Christoffel symbols} of $\nabla$}
		  	\end{displaymath}
		  }
		}
		\vfill
		\fbox{
		  \parbox{\textwidth}{
		  	Extension of operator $\nabla_X$ from $\mathfrak{X}(M)$ to $\Omega^1(M)$ is implemented by:
		  	\begin{displaymath}
		  		\left\langle \nabla_X \omega \right\vert \left. Y \right\rangle \coloneqq
		  		\nabla_X \left\langle  \omega \right\vert \left. Y \right\rangle - 
		  		\left\langle \omega \right\vert \left. \nabla_X Y \right\rangle
		  	\end{displaymath}
		  }
		}	
	\end{minipage}
		\footnote[3]{Caveat: pull-back is well defined only on covariant tensors. Otherwise $g$ has to be a diffeomorphism.}
  \end{landscape}
\end{document}