\documentclass[a4paper,12pt]{scrartcl}
\usepackage{pdflscape}
\usepackage{tabularx}
\usepackage{array}
\setlength{\extrarowheight}{6mm}
\usepackage{geometry}
 \geometry{
 a4paper,
 total={185mm,277mm},
 left=5mm,
 right=5mm,
 top=5mm,
 bottom=5mm,
 }
 
\usepackage[subpreambles=true]{standalone}
 
\usepackage{amssymb}
\usepackage[leqno]{amsmath}
\usepackage{amsfonts}

\usepackage[symbol]{footmisc}
\renewcommand*{\thefootnote}{\fnsymbol{footnote}}


\usepackage[tensors,diffgeo]{../Math-Symbols-List/toninus-math-symbols}


%https://tex.stackexchange.com/questions/48980/whole-page-table-with-tabularx

\begin{document}
  \begin{landscape}
    \thispagestyle{empty}
    \noindent
    \paragraph{CARTAN CALCULUS}
    	\mbox{}\\
        $\quad$Suppose $M$ is a smooth manifold, $x^\mu$ a coordinate chart. Denote by $\Omega(M)$ the algebra of differential forms on $M$ and by $\mathfrak{X}(M)$ the $C^\infty(M)$-module of vector fields.  \\
    \vspace{5mm}
    \begin{tabularx}{\linewidth}{|c|X|X|c|}
      \hline
     	  & $f \in C^\infty(M)$ & $\ExtD x^\mu \in \Omega^1(M)$ & $\omega^{(k)} \wedge \beta$  \\
      \hline
      	$\ExtD$ & $\ExtD f = \left(\dfrac{\partial f }{\partial x^\nu} \right) \: \ExtD x^\nu$ & 0 & $\left( \ExtD \omega \right) \wedge \beta + (-)^k \omega \wedge \left( \ExtD\beta \right) $ \\
      	$\Lie_X$ & $\Lie_X f = X(f) = \left(\dfrac{\partial f }{\partial x^\nu} \right) \: X^\nu$ & $\Lie_X \ExtD x^\mu = \ExtD \left(X^a \partial_a x^\mu\right) =  \ExtD \left(X^a \delta_a^\mu \right) = \ExtD\left(X^\mu\right) = \left(\dfrac{\partial X^\mu}{\partial x^\nu}\right)\ExtD x^\nu$ & $\left( \Lie_X \omega \right) \wedge \beta + \omega \wedge \left(\Lie_X\beta \right)$ \\
      	$\iota_X$  & $0$ & $\iota_X \ExtD x^\mu = \ExtD x^\mu (X) = X^\mu$ & $\left( \iota_X \omega \right) \wedge \beta + (-)^k \omega \wedge \left( \iota_X\beta \right) $ \\
      	$g^\ast$  \footnotemark[4]  & $g^\ast \left(f\right) = f \circ g $ & $ g^\ast \left(\ExtD x^\mu \right) = \ExtD\left(x^\mu \circ g \right)$ & $g^\ast\left(\omega\right) \wedge g^\ast \left( \beta \right)$ \\ %& & & & & \\
      \hline
    \end{tabularx}
	\begin{minipage}[c][.585\textheight]{0.46 \linewidth}
		\fbox{
		  \parbox{\textwidth}{
			    The \emph{Cartan calculus} consists of the following three \emph{graded derivations} on $\Omega(M)$
				\begin{itemize}\itemsep0em 
					\item the \emph{exterior derivative} $d$;
					\item the space of \emph{Lie derivative operators} $\Lie_X$, where $X \in \mathfrak{X}(M)$;
					\item the space of \emph{contraction operators} $\iota_X$, where $X \in \mathfrak{X}(M)$.
				\end{itemize}
				\begin{center}
						\includestandalone[scale=0.80]{cartancalculusdiagram}	
				\end{center}
				
				Together with the following identities:
				\begin{align}
					\ExtD^2 &= 0 \label{cartfirst}\\
					\ExtD \Lie_X - \Lie_X d &= 0 \\
					\ExtD \iota_X + \iota_X d &= \Lie_X \label{magic}\\
					\Lie_X \Lie_Y - \Lie_Y \Lie_X &= \Lie_{[X,Y]} \\
					\Lie_X \iota_Y - \iota_Y \Lie_X &= \iota_{[X,Y]}\\
					\iota_X \iota_Y + \iota_Y \iota_X &= 0 \label{cartlast}
				\end{align}
		  }
		}
		\vfill
	\end{minipage}
	\hspace{1cm}
	\begin{minipage}[c][.55\textheight]{0.46 \linewidth}
		\fbox{
		  \parbox{\textwidth}{
				\begin{displaymath}
			    	\Omega(M) = \left(\bigoplus_{k=0}^m \Omega^k(M), \wedge \right) \qquad \textrm{\emph{Grassmann Algebra} on M}
			    \end{displaymath}
			    is a (graded-)commutative graded algebra over the ring $C^\infty(M)$.
		  }
		}
		\vfill
		Recall:	
		\vfill	
		\fbox{
		  \parbox{\textwidth}{
			A \emph{(graded-)commutatative ($I$-)graded algebra} $(V, \wedge)$ is an algebra over ring $R$ such that :
			\begin{displaymath}
				V = \bigoplus_{i \in I} V_i
			\end{displaymath}
			and
			\begin{displaymath}
				v^{(i)} \wedge w = (-)^i w \wedge v^{(i)} \qquad \forall v^{(i)} \in V_{i}, \quad w \in V
			\end{displaymath}
		  }
		}
		\vfill			
		\fbox{
		  \parbox{\textwidth}{
				A \emph{graded derivation} of $\Omega(M)$ is a degree $k$ linear operator $A$ on $\Omega(M)$ :
				\begin{displaymath}
					A:\Omega^p(M) \rightarrow \Omega^{p+k}(M)
				\end{displaymath}
				such that:
				\begin{displaymath}
					A (\omega \wedge \eta) = A(\omega) \wedge \eta + (-1)^{kp} \omega \wedge A(\eta) \qquad \forall \omega \in \Omega^k(M) , \; \eta \in \Omega^\cdot(M)
				\end{displaymath}
		  }
		}
	\end{minipage}
		\footnote[4]{$g:N \sim (y^A) \rightarrow M \sim (x^\mu) $}
   
  \end{landscape}
\end{document}
