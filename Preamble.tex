%Per le Figure
\usepackage[english]{babel}
\usepackage{graphicx}

%simboli matematici strani quali unione disgiunta
\usepackage{amssymb}

%Scrivere Sotto i simboli
\usepackage{amsmath}

%Diagrammi Commutativi
\usepackage{tikz}
\usetikzlibrary{matrix}

%Il simbolo di Identità
\usepackage{dsfont}

%Per riflettere i simboli...
\usepackage{graphicx}


%link iNTERNET
\usepackage{hyperref}

%Enumerate with letters
\usepackage{enumerate}

%Slash over letter
\usepackage{cancel}

%Usare bibiliografia bibtex
%\bibliographystyle{plain}

%Danger sign
\usepackage{fourier}

%:=
\usepackage{mathtools}



%Common symbols
\usepackage[all]{../Math-Symbols-List/toninus-math-symbols}

%% Warning: use the ../ because the source including these files are in the upper folder! :D
%Temporaneo, Aggiunta della mia classe teorem... Deve diventare un pacchetto! 
\input{../Latex-Theorem/TheoremTemplateToninus.tex}

%  Glossario
\usepackage{glossaries}

\makenoidxglossaries

%How to:
%affinchè la voce venga printata nella lista va prima chiamata nel testo come e.g. \gls{Bundle}
%ricordarsi di chiamarlo almeno una volta così, dopo usare il command per evitare il ripetuto hyperref
% anche se si potrebbe evitare visto che il quadratino del link non dovrebbe apparire in stampa

%Advanced Differential Geometry
\newglossaryentry{Bundle}%
{%
	name={\ensuremath{E = (E,\pi , M;Q)}},
	description={ Fiber Bundles $\pi: E\rightarrow M$ with typical fiber $Q$},
    sort={B}
}

\newglossaryentry{Sections}%
{%
	name={\ensuremath{\Gamma^\infty(E)}},
	description={ Smooth sections on the bundle $E$.},
    sort={S}
}