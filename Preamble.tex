%Per le Figure
\usepackage[english]{babel}
\usepackage{graphicx}

%simboli matematici strani quali unione disgiunta
\usepackage{amssymb}

%Scrivere Sotto i simboli
\usepackage{amsmath}

%Diagrammi Commutativi
\usepackage{tikz}
\usetikzlibrary{matrix}

%Il simbolo di Identità
\usepackage{dsfont}

%Per riflettere i simboli...
\usepackage{graphicx}


%link iNTERNET
\usepackage{hyperref}

%Enumerate with letters
\usepackage{enumerate}

%Slash over letter
\usepackage{cancel}

%Usare bibiliografia bibtex
%\bibliographystyle{plain}

%Danger sign
\usepackage{fourier}

%:=
\usepackage{mathtools}



%Common symbols
%Common math symbols
	%Number fields
		\newcommand{\Real}{\mathbb{R}}
		\newcommand{\Natural}{\mathbb{N}}
		\newcommand{\Relative}{\mathbb{Z}}
		\newcommand{\Rational}{\mathbb{Q}}
		\newcommand{\Complex}{\mathbb{C}}
	
%equality lingo
	%must be equal
		\newcommand{\mbeq}{\overset{!}{=}} 

% function
	%Domain
		\newcommand{\dom}{\mathrm{dom}}
	%Range
		\newcommand{\ran}{\mathrm{ran}}
	

% Set Theory
	% Power set (insieme delle parti
		\newcommand{\PowerSet}{\mathcal{P}}

%Differential Geometry
	% Atlas
		\newcommand{\Atlas}{\mathcal{A}}
	%support
		\newcommand{\supp}{\textrm{supp}}

	
	
%Category Theory
	%Mor set http://ncatlab.org/nlab/show/morphism
%		\newcommand{\hom}{\textrm{hom}}

%Geometric Lagrangian Mechanics
	% Kinematic Configurations
		\newcommand{\Conf}{\mathtt{C}}
	%Solutions Space
		\newcommand{\Sol}{\mathtt{Sol}}
	%Lagrangian class
		\newcommand{\Lag}{\mathsf{Lag}}
	%Lagrangiana
		\newcommand{\Lagrangian}{\mathcal{L}}
	%Data
		\newcommand{\Data}{\mathsf{Data}}
	%unique solution map
		\newcommand{\SolMap}{\mathbf{s}}
		
		
		
%Peierls (per non sbagliare più)
		\newcommand{\Pei}{Peierls}

%% Warning: use the ../ because the source including these files are in the upper folder! :D
%Temporaneo, Aggiunta della mia classe teorem... Deve diventare un pacchetto! 
\input{../../Latex-Theorem/TheoremTemplateToninus.tex}

%  Glossario
\usepackage{glossaries}

\makenoidxglossaries

%How to:
%affinchè la voce venga printata nella lista va prima chiamata nel testo come e.g. \gls{Bundle}
%ricordarsi di chiamarlo almeno una volta così, dopo usare il command per evitare il ripetuto hyperref
% anche se si potrebbe evitare visto che il quadratino del link non dovrebbe apparire in stampa

%Advanced Differential Geometry
\newglossaryentry{Bundle}%
{%
	name={\ensuremath{E = (E,\pi , M;Q)}},
	description={ Fiber Bundles $\pi: E\rightarrow M$ with typical fiber $Q$},
    sort={B}
}

\newglossaryentry{Sections}%
{%
	name={\ensuremath{\Gamma^\infty(E)}},
	description={ Smooth sections on the bundle $E$.},
    sort={S}
}