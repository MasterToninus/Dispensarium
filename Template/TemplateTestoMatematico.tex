
%\documentclass[a4paper,12pt]{article}	%No Chapter
\documentclass[a4paper,12pt]{report}	%Si Chapter


\usepackage{amsmath}
\usepackage{amsthm} % dopo amsmath

%Basic theorem style
\theoremstyle{plain}
\newtheorem{thm}{Teorema}[section]
\newtheorem{cor}[thm]{Corollario}
\newtheorem{lem}[thm]{Lemma}
\newtheorem{prop}[thm]{Proposizione}

\theoremstyle{definition}
\newtheorem{defn}{Definizione}[chapter]

\theoremstyle{remark}
\newtheorem{oss}{Osservazione}




\begin{document}

Prova

\begin{thm}[titolo]
enunciato Teorema
\end{thm}

\begin{lem}[titolo]
enunciato Lemma
\end{lem}

\begin{prop}[titolo]
enunciato Proposizione
\end{prop}

\begin{cor}[titolo]
enunciato
\end{cor}

\begin{defn}[titolo]
enunciato
\end{defn}

\begin{oss}[titolo]
enunciato
\end{oss}

Cosa vorrei:
un begin che crei un riquadro con
\begin{itemize}
\item Titolo = teorema, proposizione, lemma, corollario, dimostrazione, osservazione, definizione provvisoria (mette una latterina vicino al numero), definizione della notazione, ..
\item Enunciato = facoltativo, da una descrizione a parole del teorema
\item Ipotesi (listate e numerate magari anche sottolistabili)
\item tesi
\item dimostrazione in un altro riquadro!

\end{itemize}

I miei testi matematici vorrei che fossero schematizzabili in sentence (come il mio utopico programma che non vedrà mai la luce), cercando di mimare un po' pretenziosamente la forma del Tractatus di \textbf{Wittgenstein}

ogni porzione di testo deve essere ben definita nel suo scopo, introduzione del argo, motivazione del perchè dedicarsi all'argomento, e per le cose più da fisico le giustificazioni degli assunti






\end{document}
