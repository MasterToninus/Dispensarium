%^^^^^^^^^^^^^^^^^^^^^^^^^^^^^^^^^^^^^^^^^^^^^^^^^^^^^^^^^^^^^^^^^^^^^^^^^^^^^^^
%  Section on Ideal
%_______________________________________________________________________________

%\documentclass{standalone}
\documentclass[a4paper,12pt]{scrartcl}    %attenzione! uso scratctl che permette sottotitolo

%Per le Figure
\usepackage[english]{babel}
\usepackage{graphicx}

%simboli matematici strani quali unione disgiunta
\usepackage{amssymb}

%Scrivere Sotto i simboli
\usepackage{amsmath}

%Diagrammi Commutativi
\usepackage{tikz}
\usetikzlibrary{matrix}

%Il simbolo di Identità
\usepackage{dsfont}

%Per riflettere i simboli...
\usepackage{graphicx}


%link iNTERNET
\usepackage{hyperref}

%Enumerate with letters
\usepackage{enumerate}

%Slash over letter
\usepackage{cancel}

%Usare bibiliografia bibtex
%\bibliographystyle{plain}

%Danger sign
\usepackage{fourier}

%:=
\usepackage{mathtools}



%Common symbols
%Common math symbols
	%Number fields
		\newcommand{\Real}{\mathbb{R}}
		\newcommand{\Natural}{\mathbb{N}}
		\newcommand{\Relative}{\mathbb{Z}}
		\newcommand{\Rational}{\mathbb{Q}}
		\newcommand{\Complex}{\mathbb{C}}
	
%equality lingo
	%must be equal
		\newcommand{\mbeq}{\overset{!}{=}} 

% function
	%Domain
		\newcommand{\dom}{\mathrm{dom}}
	%Range
		\newcommand{\ran}{\mathrm{ran}}
	

% Set Theory
	% Power set (insieme delle parti
		\newcommand{\PowerSet}{\mathcal{P}}

%Differential Geometry
	% Atlas
		\newcommand{\Atlas}{\mathcal{A}}
	%support
		\newcommand{\supp}{\textrm{supp}}

	
	
%Category Theory
	%Mor set http://ncatlab.org/nlab/show/morphism
%		\newcommand{\hom}{\textrm{hom}}

%Geometric Lagrangian Mechanics
	% Kinematic Configurations
		\newcommand{\Conf}{\mathtt{C}}
	%Solutions Space
		\newcommand{\Sol}{\mathtt{Sol}}
	%Lagrangian class
		\newcommand{\Lag}{\mathsf{Lag}}
	%Lagrangiana
		\newcommand{\Lagrangian}{\mathcal{L}}
	%Data
		\newcommand{\Data}{\mathsf{Data}}
	%unique solution map
		\newcommand{\SolMap}{\mathbf{s}}
		
		
		
%Peierls (per non sbagliare più)
		\newcommand{\Pei}{Peierls}


%Temporaneo, Aggiunta della mia classe teorem... Deve diventare un pacchetto!
\input{../Latex-Theorem/TheoremTemplateToninus.tex}

\usepackage{colortbl}%http://tex.stackexchange.com/questions/50349/color-only-a-cell-of-a-table
%multi-row
\usepackage{multirow}

\newcommand{\OpA}{\otimes}
\newcommand{\OpB}{\oplus}

\begin{document}
	\section{Ideal (Semigroups)}
	Consider a semigroup $(S, \OpA)$ and two subsets $A$ and $B$.
	\begin{notationfix}
		\begin{displaymath}
			A B \coloneqq \{ a \OpA b \; \vert \; a \in A ;\, b \in B \}
		\end{displaymath}
	\end{notationfix}
	\begin{notationfix}
		Left coset of $a \in A\subset S$:
		\begin{displaymath}
			[a]_L = a S
		\end{displaymath}
		Right coset of $a \in A\subset S$:
		\begin{displaymath}
			[a]_R =  S a
		\end{displaymath}
	\end{notationfix}
	
	\begin{definition}[SubSemigroup]
		$A$ is a \emph{semigroup of} $S$ iff:
		\begin{displaymath}
			A A \subset A 
		\end{displaymath}
	\end{definition}
	
		\begin{definition}[Normal SubSemigroup]
		$A$ is a \emph{normal semigroup of} $S$, $A \triangleleft S$ iff:
		\begin{displaymath}
			a S = S a \qquad \forall a \in A
		\end{displaymath}
	\end{definition}
	\begin{Warning}
		Con i gruppi si legge come coniugazione \url{https://en.wikipedia.org/wiki/Normal_subgroup}
	\end{Warning}
	

	\begin{definition}[Right Ideal]
		$A$ is a \emph{right ideal of} $S$ iff:
		\begin{displaymath}
			S A \subset A 
		\end{displaymath}
	\end{definition}
	
	\begin{definition}[Left Ideal]
		$A$ is a \emph{left ideal of} $S$ iff:
		\begin{displaymath}
			S A \subset A 
		\end{displaymath}
	\end{definition}
	
	\begin{definition}[Two-sided (Bilateral) Ideal]
		$A$ is a \emph{bilateral ideal of} $S$ if it is both a left and a right ideal.
		\textit{i.e.:}
		\begin{displaymath}
			S A S \subset A 
		\end{displaymath}
	\end{definition}
	
	\begin{proposition}
		If $S$ is a monoid (unital semigroup) and $A \subset S$ a submonoid, then	
		\begin{displaymath}
			  S A S \subset A \qquad \Leftrightarrow \qquad A\textrm{ two-sided ideal}
		\end{displaymath}
	\end{proposition}
	\begin{proof}
	Existence of unital element in $S$ guarantees that both $S A= S A {1}$ and $A S = {1} A S $ are subsets of $S A S$.\\	
	The converse hold true for every semigroups.
	\end{proof}

	\section{Ideal (Rings)}
	Consider an arbitrary ring $(R,\OpB,\OpA)$, let $(R,\OpB)$ be its \emph{additive group}.
	\begin{definition}[Right Ideal]
		Subset $I \subset R$ such that:
		\begin{enumerate}
			\item $(I, \OpB)$ is an additive subgroup of $(R, \OpB)$.
			\item $I$ absorbs multiplication on the right:
				\begin{displaymath}
					( x \OpA r ) \in I \qquad \forall  x \in I, \forall r \in R
				\end{displaymath}
		\end{enumerate}
	\end{definition}

	\begin{definition}[Left Ideal]
		Subset $I \subset R$ such that:
		\begin{enumerate}
			\item $(I, \OpB)$ is an additive subgroup of $(R, \OpB)$.
			\item $I$ absorbs multiplication on the left:
				\begin{displaymath}
					( r \OpA x )\in I \qquad \forall  x \in I, \forall r \in R
				\end{displaymath}
		\end{enumerate}
	\end{definition}

	\begin{definition}[Two-sided (Bilateral) Ideal]
		$A$ is a \emph{bilateral ideal of} $S$ if it is both a left and a right ideal.
	\end{definition}

	Consider a generic subset $I' \subset R$:
	\begin{definition}[Generated (Two-sided) Ideal]
		The two-sided ideal generated by $I'$ is the ideal:
		\begin{displaymath}
			I = \textrm{span} \left( a \OpA x \OpA b \in A \; \vert \; a,b\in R ; \: x \in I' \right)
		\end{displaymath}
		\begin{Warning}
			Span non è termine giusto! \url{https://en.wikipedia.org/wiki/Ideal_(ring_theory)#Ideal_generated_by_a_set}
		\end{Warning}
	
	
	\end{definition}


	\begin{Warning}
		...
	\end{Warning}
	
	Consider an arbitrary ring $(R,\OpB,\OpA)$ and a two-sided ideal $I\subset R$:
	\begin{proposition}[ $a$ congruent $b$ modulo $I$]
		Consider $a,b \in R$:
		\begin{displaymath}
			a \sim b \qquad \Leftrightarrow (a  - b ) \in I
		\end{displaymath}
		is a \emph{Congruence Relation}
	\end{proposition}
	\url{https://en.wikipedia.org/wiki/Quotient_ring}
	
	\begin{definition}[Ring quotient by an ideal]
		Is the set 
		\begin{displaymath}
		 R/I \coloneqq \{ [a]\}
		\end{displaymath}
		where:
		\begin{displaymath}
			[a] = a \OpB I = \{ a + r \: \vert \: r \in I \}
		\end{displaymath}
		endowed with operations:
		\begin{itemize}
			\item $ (a + I) + (b + I) = (a + b) + I $
			\item $ (a + I)(b + I) = (a b) + I$
		\end{itemize}

	
	\end{definition}

\end{document}